\pagebreak
\chapter{Thesis Software}
\label{appndx:code}

The following software can also be found in the up-to-date version on \href{https://gitlab.com/ttis-simulations/ttis-software/-/tree/master/Scripts}{GitLab}.

\section{Meshing}
\begin{lstlisting}[language=Python,caption={Class handling electrode operations on the models},captionpos=b,label=lst:electrode_operations_class]
import pymesh
import numpy as np

class ElectrodeOperations:
    def __init__(self, surface_mesh: pymesh.Mesh, electrode_attributes: dict):
        self.surface_mesh = surface_mesh
        self.surface_mesh.enable_connectivity() # Required by some functions below
        self.electrode_attributes = electrode_attributes
        self.electrode_array = {}

    def add_electrodes_on_skin(self):
        if not self.electrode_array:
            raise AttributeError('The electrodes shall be positioned first before added on the surface. Please call the positioning function first.')
        electrode_mesh = self.get_electrode_single_mesh()
        # Get the surface outline including the electrode
        model = pymesh.merge_meshes((self.surface_mesh, electrode_mesh))
        outer_hull = pymesh.compute_outer_hull(model)

        # Create the surface with the electrode mesh imprinted
        electrode_tan_mesh = pymesh.boolean(electrode_mesh, self.surface_mesh, 'difference')
        outer_diff = pymesh.boolean(outer_hull, electrode_tan_mesh, 'difference')
        conditioned_surface = pymesh.merge_meshes((outer_diff, electrode_tan_mesh))

        # Generate the surface with the electrode on
        face_id = np.arange(conditioned_surface.num_faces)
        conditioned_surface = pymesh.remove_duplicated_vertices(conditioned_surface)[0] # Remove any duplicate vertices

        return [pymesh.submesh(conditioned_surface, np.isin(face_id, pymesh.detect_self_intersection(conditioned_surface)[:, 0], invert=True), 0), outer_diff]  # Get rid of the duplicate faces on the tangent surface, without merging the points

    def standard_electrode_positioning(self, width=4, radius=4, elements=150):
        closest_point = pymesh.distance_to_mesh(self.surface_mesh, self.electrode_attributes['coordinates'])[1] # Get the closest point to the one provided

        i = 0
        for electrode_name in self.electrode_attributes['names']:
            p_i = self.surface_mesh.vertices[self.surface_mesh.faces[closest_point[i]]][0] # Get the surface vertex coordinates
            electrode_orientation = self.__orient_electrode(p_i) # Orient the electrode perpendicular to the surface

            # Generate the electrode cylinder and save to the output dictionary
            electrode_cylinder = pymesh.generate_cylinder(p_i - (width * electrode_orientation)/4., p_i + (width * electrode_orientation)/4., radius, radius, elements)
            self.electrode_array[electrode_name] = electrode_cylinder
            i = i + 1

    def get_electrode_array(self):
        if not self.electrode_array:
            raise AttributeError('Electrodes are not positioned yet. Please call the positioning function.')
        return self.electrode_array

    def get_electrode_single_mesh(self):
        if not self.electrode_array:
            raise AttributeError('Electrodes are not positioned yet. Please call the positioning function.')
        return pymesh.merge_meshes((e_mesh for e_mesh in self.electrode_array.values()))

    def __orient_electrode(self, init_point):
        point_id = np.where(np.sum(self.surface_mesh.vertices == init_point, axis=1))[0][0] # Unique point assumed
        face = self.surface_mesh.get_vertex_adjacent_faces(point_id)[0]

        points = []
        for point in self.surface_mesh.vertices[self.surface_mesh.faces[face]]:
            if np.sum(point != init_point):
                points.append(point)
        p_1 = points[0] - init_point
        p_2 = points[1] - init_point

        normal = np.cross(p_1, p_2)
        return normal/np.linalg.norm(normal)


\end{lstlisting}

\begin{lstlisting}[language=Python,caption={Class handling all mesh operations for the models},captionpos=b,label=lst:mesh_operations_class]
import os
import meshio
import pymesh
import numpy as np

import Meshing.ElectrodeOperations as ElecOps
import FileOps.FileOperations as FileOps

class MeshOperations(ElecOps.ElectrodeOperations):
    def __init__(self):
        self.merged_meshes = None
        print()

    def phm_model_meshing(self, base_path: str, suffix_name: str, mesh_filename: str, electrode_attributes: dict):
        ##### Import th model files to create the mesh
        print("Importing files") # INFO log
        skin_stl = pymesh.load_mesh(os.path.join(base_path, 'skin' + suffix_name))
        skull_stl = pymesh.load_mesh(os.path.join(base_path, 'skull' + suffix_name))
        csf_stl = pymesh.load_mesh(os.path.join(base_path, 'csf' + suffix_name))
        gm_stl = pymesh.load_mesh(os.path.join(base_path, 'gm' + suffix_name))
        wm_stl = pymesh.load_mesh(os.path.join(base_path, 'wm' + suffix_name))
        ventricles_stl = pymesh.load_mesh(os.path.join(base_path, 'ventricles' + suffix_name))
        cerebellum_stl = pymesh.load_mesh(os.path.join(base_path, 'cerebellum' + suffix_name))
        ##### Import th model files to create the mesh

        ##### Electrode placement
        print("Placing Electrodes") # INFO log
        electrodes_object = ElecOps.ElectrodeOperations(skin_stl, electrode_attributes)
        electrodes_object.standard_electrode_positioning(width=electrode_attributes['width'], radius=electrode_attributes['radius'], elements=electrode_attributes['elements'])
        skin_with_electrodes = electrodes_object.add_electrodes_on_skin()[0]

        electrodes_single_mesh = electrodes_object.get_electrode_single_mesh()
        electrodes = pymesh.boolean(electrodes_single_mesh, skin_stl, 'difference')
        pymesh.map_face_attribute(electrodes_single_mesh, electrodes, 'face_sources')
        ##### Electrode placement

        self.merged_meshes = pymesh.merge_meshes((skin_with_electrodes, skull_stl, csf_stl, gm_stl, wm_stl, cerebellum_stl, ventricles_stl))
        regions = self.region_points([skin_stl, skull_stl, csf_stl, gm_stl, wm_stl, cerebellum_stl, ventricles_stl], electrodes, 0.1)

        FileOps.FileOperations.poly_write(mesh_filename, self.merged_meshes.vertices, self.merged_meshes.faces, self.merged_meshes.get_attribute('face_sources'), regions)

    def region_points(self, boundary_surfaces: list, electrode_mesh, shift: float, max_volume=30):
        points = {}
        i = 1
        for surface in boundary_surfaces:
            vertices = surface.vertices
            max_z_point_id = np.where(vertices[:, 2] == np.amax(vertices[:, 2]))[0][0]
            points[str(i)] = {
                'coordinates': np.round(vertices[max_z_point_id] - np.absolute(np.multiply(vertices[max_z_point_id]/np.linalg.norm(vertices[max_z_point_id]), [0, 0, shift])), 6),
                'max_volume': max_volume,
            }
            i = i + 1

        i = 10
        electrode_regions = electrode_mesh.get_attribute('face_sources')
        electrode_mesh.add_attribute('vertex_valance')
        vertex_valance = electrode_mesh.get_attribute('vertex_valance')
        for region in np.unique(electrode_regions):
            faces = electrode_mesh.faces[np.where(electrode_regions == region)[0]]
            vertices = electrode_mesh.vertices[np.unique(faces)]
            max_valance_point = np.where(vertex_valance[np.unique(faces)] == np.amax(vertex_valance[np.unique(faces)]))[0][0]
            points[str(i)] = {
                'coordinates': np.round(vertices[max_valance_point] - np.multiply(vertices[max_valance_point]/np.linalg.norm(vertices[max_valance_point]), shift), 6),
                'max_volume': max_volume,
            }
            i = i + 1

        return points
\end{lstlisting}

\section{FEM}

\begin{lstlisting}[language=Python,caption={Class interfacing to SfePy and contains routines for the PHM and spherical model \gls{FEM} solution.},captionpos=b,label=lst:fem_solver_class]
from __future__ import absolute_import
import os
import gc
import sys
import meshio
import numpy as np

#### SfePy libraries
from sfepy.base.base import Struct
from sfepy.discrete import (FieldVariable, Material, Integral, Equation, Equations, Problem, Function)
from sfepy.discrete.fem import Mesh, FEDomain, Field
from sfepy.terms import Term
from sfepy.discrete.conditions import Conditions, EssentialBC
from sfepy.solvers.ls import PETScKrylovSolver
from sfepy.solvers.nls import Newton
#### SfePy libraries

import Meshing.modulation_envelope as mod_env

class Solver:
    def __init__(self, settings_file: dict, settings_header: str, electrode_system: str):
        self.__settings = settings_file
        self.__settings_header = settings_header
        if os.name == 'nt':
            self.__extra_path = '_windows'
        else:
            self.__extra_path = ''
        sys.path.append(os.path.realpath(settings_file[settings_header]['lib_path' + self.__extra_path]))

        self.__linear_solver = None
        self.__non_linear_solver = None
        self.__overall_volume = None
        self.conductivities = {}
        self.electrode_system = electrode_system

        # Read from settings
        self.__material_conductivity = None
        self.__selected_model = None
        self.domain = None
        self.problem = None
        self.essential_boundaries = []
        self.field_variables = {}
        self.fields = {}

    def load_mesh(self, model=None, connectivity='3_4'):
        if model is None:
            raise AttributeError('No model was selected.')
        mesh = meshio.read(self.__settings[self.__settings_header][model]['mesh_file' + self.__extra_path])
        self.__selected_model = model

        vertices = mesh.points
        vertex_groups = np.empty(vertices.shape[0])
        cells = mesh.cells[0][1]
        cell_groups = mesh.cell_data['cell_scalars'][0]

        for group in np.unique(cell_groups):
            roi_cells = np.unique(cells[np.where(cell_groups == group)[0]])
            vertex_groups[roi_cells] = group

        loaded_mesh = Mesh.from_data('mesh_name', vertices, vertex_groups, [cells], [cell_groups], [connectivity])
        self.domain = FEDomain('domain', loaded_mesh)

    def define_field_variable(self, var_name: str, field_name: str):
        # TODO: Check if the provided field exists
        if not self.__overall_volume:
            self.__assign_regions()
        if field_name not in self.fields.keys():
            self.fields[field_name] = Field.from_args(field_name, dtype=np.float64, shape=(1, ), region=self.__overall_volume, approx_order=1)

        self.field_variables[var_name] = {
            'unknown': FieldVariable(var_name, 'unknown', field=self.fields[field_name]),
            'test': FieldVariable(var_name + '_test', 'test', field=self.fields[field_name], primary_var_name=var_name),
        }

    def define_essential_boundary(self, region_name: str, group_id: int, field_variable: str, field_value: float):
        # TODO: Add a check to see if the provided potential variable is a defined potential
        # TODO: Do not run if there are no field variables
        if field_variable not in self.field_variables.keys():
            raise AttributeError('The field variable {}')
        temporary_domain = self.domain.create_region(region_name, 'vertices of group ' + str(group_id), 'facet', add_to_regions=False)
        self.essential_boundaries.append(EssentialBC(region_name, temporary_domain, {field_variable + '.all' : field_value}))

    def solver_setup(self, max_iterations=250, relative_tol=1e-7, absolute_tol=1e-3, verbose=False):
        self.__linear_solver = PETScKrylovSolver({
            'ksp_max_it': max_iterations,
            'ksp_rtol': relative_tol,
            'ksp_atol': absolute_tol,
            'ksp_type': 'cg',
            'pc_type': 'hypre',
            'pc_hypre_type': 'boomeramg',
            'pc_hypre_boomeramg_coarsen_type': 'HMIS',
            'verbose': 2 if verbose else 0,
        })

        self.__non_linear_solver = Newton({
            'i_max': 1,
            'eps_a': absolute_tol,
        }, lin_solver=self.__linear_solver)

    def run_solver(self, save_results: bool, post_process_calculation=True, output_dir=None, output_file_name=None):
        if not self.__non_linear_solver:
            raise AttributeError('The solver is not setup. Please set it up before calling run.')
        self.__material_definition()

        self.problem = Problem('temporal_interference', equations=self.__generate_equations())
        self.problem.set_bcs(ebcs=Conditions(self.essential_boundaries))
        self.problem.set_solver(self.__non_linear_solver)
        self.problem.setup_output(output_filename_trunk=output_file_name, output_dir=output_dir)

        if post_process_calculation:
            return self.problem.solve(post_process_hook=self.__post_process, save_results=save_results)
        return self.problem.solve(save_results=save_results)

    def set_custom_post_process(self, function):
        self.__post_process = function

    def clear_all(self):
        del self.domain
        del self.__overall_volume
        del self.essential_boundaries
        del self.field_variables
        del self.fields
        del self.problem
        gc.collect()

    def __generate_equations(self):
        # TODO: Add a check for the existence of the fields
        integral = Integral('i1', order=2)

        equations_list = []
        for field_variable in self.field_variables.items():
            term_arguments = {
                'conductivity': self.__material_conductivity,
                field_variable[0] + '_test': field_variable[1]['test'],
                field_variable[0]: field_variable[1]['unknown']
            }
            equation_term = Term.new('dw_laplace(conductivity.val, ' + field_variable[0] + '_test, ' + field_variable[0] + ')', integral, self.__overall_volume, **term_arguments)
            equations_list.append(Equation('balance', equation_term))

        return Equations(equations_list)

    def __material_definition(self):
        if not self.conductivities:
            self.__assign_regions()
        self.__material_conductivity = Material('conductivity', function=Function('get_conductivity', lambda ts, coors, mode=None, equations=None, term=None, problem=None, **kwargs: self.__get_conductivity(ts, coors, mode, equations, term, problem, conductivities=self.conductivities)))

    def __assign_regions(self):
        self.__overall_volume = self.domain.create_region('Omega', 'all')

        for region in self.__settings[self.__settings_header][self.__selected_model]['regions'].items():
            self.domain.create_region(region[0], 'cells of group ' + str(region[1]['id']))
            self.conductivities[region[0]] = region[1]['conductivity']

        for electrode in self.__settings[self.__settings_header]['electrodes'][self.electrode_system].items():
            self.domain.create_region(electrode[0], 'cells of group ' + str(electrode[1]['id']))
            self.conductivities[electrode[0]] = self.__settings[self.__settings_header]['electrodes']['conductivity']

    def __get_conductivity(self, ts, coors, mode=None, equations=None, term=None, problem=None, conductivities=None):
        # Execute only once at the initialization
        if mode == 'qp':
            values = np.empty(int(coors.shape[0]/4)) # Each element corresponds to one coordinate of the respective tetrahedral edge

            # Save the conductivity values
            for domain in problem.domain.regions:
                if domain.name in conductivities.keys():
                    values[domain.entities[3]] = conductivities[domain.name]

            values = np.repeat(values, 4) # Account for the tetrahedral edges
            values.shape = (coors.shape[0], 1, 1)

            return {'val' : values}

    def __post_process(self, out, problem, state, extend=False):
        e_field_base = problem.evaluate('-ev_grad.2.Omega(potential_base)', mode='qp')
        e_field_df = problem.evaluate('-ev_grad.2.Omega(potential_df)', mode='qp')

        # Calculate the maximum modulation envelope
        modulation_cells = mod_env.modulation_envelope(e_field_base[:, 0, :, 0], e_field_df[:, 0, :, 0])
        modulation_cells = np.repeat(modulation_cells, 4, axis=0).reshape((e_field_base.shape[0], 4, 1, 1))
        modulation_points = modulation_cells.flatten()[np.unique(problem.domain.mesh.get_conn('3_4').flatten(), return_index=True)[1]]

        # Calculate the directional modulation envelope
        modulation_x = mod_env.modulation_envelope(e_field_base[:, 0, :, 0], e_field_df[:, 0, :, 0], dir_vector=[1, 0, 0])
        modulation_y = mod_env.modulation_envelope(e_field_base[:, 0, :, 0], e_field_df[:, 0, :, 0], dir_vector=[0, 1, 0])
        modulation_z = mod_env.modulation_envelope(e_field_base[:, 0, :, 0], e_field_df[:, 0, :, 0], dir_vector=[0, 0, 1])

        modulation_x = np.repeat(modulation_x, 4, axis=0).reshape((e_field_base.shape[0], 4, 1, 1))
        modulation_y = np.repeat(modulation_y, 4, axis=0).reshape((e_field_base.shape[0], 4, 1, 1))
        modulation_z = np.repeat(modulation_z, 4, axis=0).reshape((e_field_base.shape[0], 4, 1, 1))

        # Save the output
        out['e_field_base'] = Struct(name='e_field_base', mode='cell', data=e_field_base, dofs=None)
        out['e_field_df'] = Struct(name='e_field_df', mode='cell', data=e_field_df, dofs=None)
        out['max_modulation'] = Struct(name='max_modulation', mode='cell', data=modulation_cells, dofs=None)
        out['max_modulation_pts'] = Struct(name='max_modulation_pts', mode='vertex', data=modulation_points, dofs=None)
        out['modulation_x'] = Struct(name='modulation_x', mode='cell', data=modulation_x, dofs=None)
        out['modulation_y'] = Struct(name='modulation_y', mode='cell', data=modulation_y, dofs=None)
        out['modulation_z'] = Struct(name='modulation_z', mode='cell', data=modulation_z, dofs=None)

        return out
\end{lstlisting}

\section{Utilities}
\begin{lstlisting}[language=Python,caption={Class defining parsers and writers for commonly used file types. Here .gmsh and .poly are covered.},captionpos=b,label=lst:file_ops_class]
import numpy as np

class FileOperations():
    def __init__(self):
        print("here")

    @staticmethod
    def poly_write(file_name: str, nodes, faces, boundaries, regions: dict):
        with open(file_name, "wb") as m_file:
            # Nodes
            m_file.write("{} 3\n".format(len(nodes)).encode("utf-8"))
            for index in range(0, len(nodes)):
                m_file.write("{} {} {} {}\n".format(index + 1, *nodes[index]).encode("utf-8"))

            # Faces
            m_file.write("{} 1\n".format(len(faces)).encode("utf-8"))
            for index in range(0, len(faces)):
                m_file.write("1 0 {}\n".format(int(boundaries[index] + 1)).encode("utf-8"))
                m_file.write("3 {} {} {}\n".format(*faces[index] + 1).encode("utf-8"))

            # Holes NOT IMPLEMENTED YET
            m_file.write("0\n".encode("utf-8"))

            # Regions
            m_file.write("{}\n".format(len(list(regions.keys()))).encode("utf-8"))
            for region_id in regions.keys():
                m_file.write("{} {} {} {} {} {}\n".format(region_id, *regions[region_id]['coordinates'], region_id, regions[region_id]['max_volume']).encode("utf-8"))

    @staticmethod
    def gmsh_write(file_name: str, surfaces, domains, physical_tags, bounding_surface_tag):
        c_size_t = np.dtype("P")

        with open(file_name, "wb") as m_file:
            # MeshFormat (Check)
            m_file.write(b"$MeshFormat\n")
            m_file.write("4.1 {} {}\n".format(1, c_size_t.itemsize).encode("utf-8"))

            m_file.write(b"$EndMeshFormat\n")

            # PhysicalNames (Check)
            m_file.write(b"$PhysicalNames\n")
            m_file.write("{}\n".format(len(physical_tags)).encode("utf-8"))
            for physical_tag in physical_tags:
                m_file.write('3 {} "{}"\n'.format(*physical_tag).encode("utf-8"))
            m_file.write(b"$EndPhysicalNames\n")

            # Entities (Check)
            m_file.write(b"$Entities\n")
            
            m_file.write("0 0 {} {}\n".format(len(surfaces), len(domains)).encode("utf-8"))
            for i in range(0, len(surfaces)):
                m_file.write("{} {} {} {} {} {} {} {} {}\n".format(i + 1, np.amin(surfaces[i].vertices[:, 0]), np.amin(surfaces[i].vertices[:, 1]), np.amin(surfaces[i].vertices[:, 2]), np.amax(surfaces[i].vertices[:, 0]), np.amax(surfaces[i].vertices[:, 1]), np.amax(surfaces[i].vertices[:, 2]), 0, 0).encode("utf-8"))
            for i in range(0, len(domains)):
                m_file.write("{} {} {} {} {} {} {} {} {} {} {}\n".format(i + 1, np.amin(domains[i].vertices[:, 0]), np.amin(domains[i].vertices[:, 1]), np.amin(domains[i].vertices[:, 2]), np.amax(domains[i].vertices[:, 0]), np.amax(domains[i].vertices[:, 1]), np.amax(domains[i].vertices[:, 2]), 1, physical_tags[i][0], 1, bounding_surface_tag[i]).encode("utf-8"))
            m_file.write(b"$EndEntities\n")

            # Nodes (Checked?)
            float_fmt=".16e"

            surfs = 0
            voxels = 0
            for surf in surfaces:
                surfs = surfs + surf.num_vertices
            for domain in domains:
                voxels = voxels + domain.num_vertices

            m_file.write(b"$Nodes\n")
            num_blocks = len(surfaces) + len(domains)
            min_tag = 1
            max_tag = surfs + voxels

            m_file.write("{} {} {} {}\n".format(num_blocks, surfs + voxels, min_tag, max_tag).encode("utf-8"))

            ta = 1
            face_elements = []
            for i in range(0, len(surfaces)):
                # dim_entity, is_parametric
                num_elem = surfaces[i].num_vertices

                m_file.write("{} {} {} {}\n".format(2, i + 1, 0, num_elem).encode("utf-8"))
                np.arange(ta, ta + num_elem).tofile(m_file, "\n", "%d")
                m_file.write(b"\n")
                np.savetxt(m_file, surfaces[i].vertices, delimiter=" ", fmt="%" + float_fmt)
                #face_elements.append(surfaces[i].faces + 1 + (ta - 1))
                face_elements.append(surfaces[i].faces.astype(c_size_t) + 1 + (ta - 1))
                ta = ta + num_elem
                m_file.write(b"\n")

            voxel_elements = []
            for i in range(0, len(domains)):
                num_elem = domains[i].num_vertices

                m_file.write("{} {} {} {}\n".format(3, i + 1, 0, num_elem).encode("utf-8"))
                np.arange(ta, ta + num_elem).tofile(m_file, "\n", "%d")
                m_file.write(b"\n")
                np.savetxt(m_file, domains[i].vertices, delimiter=" ", fmt="%" + float_fmt)
                #voxel_elements.append(domains[i].voxels + 1 + (ta - 1))
                voxel_elements.append(domains[i].voxels.astype(c_size_t) + 1 + (ta - 1))
                ta = ta + num_elem
                m_file.write(b"\n")
            m_file.write(b"$EndNodes\n")

            # Elements
            surfs = 0
            voxels = 0
            for surf in surfaces:
                surfs = surfs + surf.num_faces
            for domain in domains:
                voxels = voxels + domain.num_voxels

            m_file.write(b"$Elements\n")
            num_blocks = len(face_elements) + len(voxel_elements)
            min_tag = 1
            max_tag = surfs + voxels

            m_file.write("{} {} {} {}\n".format(num_blocks, surfs + voxels, min_tag, max_tag).encode("utf-8"))

            ta = 1
            for i in range(0, len(face_elements)):
                num_elem = face_elements[i].shape[0]

                m_file.write("{} {} {} {}\n".format(2, i + 1, 2, num_elem).encode("utf-8"))
                np.savetxt(m_file, np.column_stack([np.arange(ta, ta + num_elem), face_elements[i]]), "%d", " ")
                ta = ta + num_elem
                m_file.write(b"\n")

            for i in range(0, len(voxel_elements)):
                num_elem = voxel_elements[i].shape[0]

                m_file.write("{} {} {} {}\n".format(3, i + 1, 4, num_elem).encode("utf-8"))
                np.savetxt(m_file, np.column_stack([np.arange(ta, ta + num_elem), voxel_elements[i]]), "%d", " ")
                ta = ta + num_elem
                m_file.write(b"\n")
            m_file.write(b"$EndElements\n")
\end{lstlisting}

\begin{lstlisting}[language=yaml,caption={Sample settings file},captionpos=b,label=lst:settings_file]
---
SfePy:
  lib_path: 'path to the scripts folder on Linux'
  lib_path_windows: 'path to the scripts folder on Windows'
  electrodes:
    conductivity: 59600000.0
    10-20:
      Fp1:
        id: 10
      Fp2:
        id: 11
      F7:
        id: 12
      F3:
        id: 13
      Fz:
        id: 14
      F4:
        id: 15
      F8:
        id: 16
      T7:
        id: 17
      C3:
        id: 18
      Cz:
        id: 19
      C4:
        id: 20
      T8:
        id: 21
      P7:
        id: 22
      P3:
        id: 23
      Pz:
        id: 24
      P4:
        id: 25
      P8:
        id: 26
      O1:
        id: 27
      O2:
        id: 28
    10-10:
      Nz:
        id: 10
      N1:
        id: 11
      Fp1:    
        id: 12
      Fpz:    
        id: 13
      Fp2:    
        id: 14
      N2:     
        id: 15
      AF9:    
        id: 16
      AF7:    
        id: 17
      AF3:    
        id: 18
      AFz:    
        id: 19
      AF4:    
        id: 20
      AF8:    
        id: 21
      AF10:   
        id: 22
      F9:     
        id: 23
      F7:     
        id: 24
      F5:     
        id: 25
      F3:     
        id: 26
      F1:     
        id: 27
      Fz:     
        id: 28
      F2:     
        id: 29
      F4:     
        id: 30
      F6:     
        id: 31
      F8:     
        id: 32
      F10:    
        id: 33
      FT9:    
        id: 34
      FT7:    
        id: 35
      FC5:    
        id: 36
      FC3:    
        id: 37
      FC1:    
        id: 38
      FCz:    
        id: 39
      FC2:    
        id: 40
      FC4:    
        id: 41
      FC6:    
        id: 42
      FT8:    
        id: 43
      FT10:   
        id: 44
      T9(M1): 
        id: 45
      T7:     
        id: 46
      C5:     
        id: 47
      C3:     
        id: 48
      C1:     
        id: 49
      Cz:     
        id: 50
      C2:     
        id: 51
      C4:     
        id: 52
      C6:     
        id: 53
      T8:     
        id: 54
      T10(M2):
        id: 55
      TP9:    
        id: 56
      TP7:    
        id: 57
      CP5:    
        id: 58
      CP3:    
        id: 59
      CP1:    
        id: 60
      CPz:    
        id: 61
      CP2:    
        id: 62
      CP4:    
        id: 63
      CP6:    
        id: 64
      TP8:    
        id: 65
      TP10:   
        id: 66
      P9:     
        id: 67
      P7:     
        id: 68
      P5:     
        id: 69
      P3:     
        id: 70
      P1:     
        id: 71
      Pz:     
        id: 72
      P2:     
        id: 73
      P4:     
        id: 74
      P6:     
        id: 75
      P8:     
        id: 76
      P10:    
        id: 77
      PO9:    
        id: 78
      PO7:    
        id: 79
      PO3:    
        id: 80
      POz:    
        id: 81
      PO4:    
        id: 82
      PO8:    
        id: 83
      PO10:   
        id: 84
      I1:     
        id: 85
      O1:     
        id: 86
      Oz:     
        id: 87
      O2:     
        id: 88
      I2:     
        id: 89
      Iz:     
        id: 90
    sphere:
      B_VCC: 
        id: 12
      B_GND: 
        id: 13
      D_VCC: 
        id: 10
      D_GND: 
        id: 11
  real_brain:
    mesh_file: 'mesh file to be loaded'
    output_dir: 'desired output directory'
    mesh_file_windows: 'mesh file to be loaded'
    output_dir_windows: 'desired output directory'
    regions:
      Skin:
        id: 1
        conductivity: 0.17
      Skull:
        id: 2
        conductivity: 0.003504
      CSF:
        id: 3
        conductivity: 1.776
      GM:
        id: 4
        conductivity: 0.2391
      WM:
        id: 5
        conductivity: 0.2651
      Cerebellum:
        id: 6
        conductivity: 0.6597
      Ventricles:
        id: 7
        conductivity: 1.776
  sphere:
    mesh_file: 'mesh file to be loaded'
    output_dir: 'desired output directory'
    mesh_file_windows: 'mesh file to be loaded'
    output_dir_windows: 'desired output directory'
    regions:
      Skin: 
        id: 1
        conductivity: 0.17
      Skull: 
        id: 2
        conductivity: 0.003504
      CSF: 
        id: 3
        conductivity: 1.776
      Brain: 
        id: 4
        conductivity: 0.234
---
\end{lstlisting}