\section{Results and tools developed}

\subsection{Open Python framework for FE simulations}
\begin{frame}
\frametitle{Open Python framework for FE simulations}
\begin{itemize}
    \uncover<1->{\item First simulations done using Sim4Life}
    \uncover<2->{\item An open-source modular framework has been developed in Python (licensed under GPLv3)}
    \uncover<3->{\item Based on SfePy \cite{Cimrman2019}, PyMesh \cite{pymesh} and TetGen \cite{tetgen} packages}
    \uncover<4->{\item Allows to solve virtually any FE problem, thus it is scalable to other fields too}
    \uncover<5->{\item Validated through the comparison with the Grossman et al. \cite{Grossman2017} results}
\end{itemize}
\end{frame}

\begin{frame}
\frametitle{Open Python framework for FE simulations}
    \only<1>{
\begin{itemize}
    \item Validated through the comparison with the Grossman et al. \cite{Grossman2017} results
\end{itemize}
% \addtocounter{figure}{1}
    \begin{figure}
        \centering
        \begin{subfigure}[b]{0.45\textwidth}
            \centering
            \includegraphics[width = \textwidth]{../assets/images/grossman_sphere_electric_field}
        \end{subfigure}
        \begin{subfigure}[b]{0.45\textwidth}
            \centering
            \includegraphics[width = \textwidth]{../assets/images/grosman_benchmark.png}
        \end{subfigure}
        \caption{Modulation envelope on the y-direction for the spherical layered model, same as on Grossman et al.}
    \end{figure}
    }
\end{frame}

\subsection{Study of tTIS on human models}
\begin{frame}
\frametitle{Study of tTIS on human models}
\begin{itemize}
    \only<1-3>{\item Variation of the electric field pattern was observed, using the same 10-20 system electrodes, across different PHM models \cite{Lee2016,Lee2018,ErikG.Lee2016}}
    \only<4>{\item Axis reference for Figure 5 and 6}
\end{itemize}
\only<1>{
\begin{figure}
    \centering
    \includegraphics[height=0.62\textheight]{assets/brain_sweep.png}
    \caption{tTIS pattern across different PHM models}
\end{figure}
}
\only<2>{
\addtocounter{figure}{1}
\begin{figure}
    \centering
    \includegraphics[height=0.55\textheight]{assets/y_axis_line.pdf}
    \caption{Electric field across the y-axis line of different PHM models. The number after the dash is the standard deviation for each model.}
    \label{fig:y_axis}
\end{figure}
}
\only<3>{
\addtocounter{figure}{2}
\begin{figure}
    \centering
    \includegraphics[height=0.55\textheight]{assets/center_of_bounds_line.pdf}
    \caption{Electric field across the center of bounds of different PHM models. The number after the dash is the standard deviation for each model.}
    \label{fig:center_of_bounds}
\end{figure}
}
\only<4>{
\addtocounter{figure}{3}
\begin{figure}
    \centering
    \begin{subfigure}[b]{0.49\textwidth}
        \centering
        \includegraphics[width=\textwidth]{assets/y_axis_line_brain.png}
    \end{subfigure}
    \begin{subfigure}[b]{0.49\textwidth}
        \centering
        \includegraphics[width=\textwidth]{assets/center_of_bounds_line_brain.png}
    \end{subfigure}
    \caption{Electric field across the center of bounds of different PHM models. The number after the dash is the standard deviation for each model.}
\end{figure}
}
\end{frame}
