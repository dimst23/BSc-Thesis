\thispagestyle{plain}
\vspace*{\fill}
\begin{center}
    \LARGE
    \textit{\textbf{Abstract}}
        
    \vspace{0.4cm}
    \large
    \textbf{Investigating non-invasive deep brain stimulation using temporally interfering electric fields}
        
    \vspace{0.4cm}
    Dimitrios Stoupis
\end{center}
\normalsize

\vspace{0.9cm}
Treating brain diseases or suppressing their symptoms is not a trivial task. Most conditions affect areas deep inside the brain, thus requiring deep brain stimulation techniques. Various invasive methods exist that inherently pose a significant risk for the overall health of the individual, one of the main reasons that there is a lot of ongoing research activity in the area of non-invasive methods. One drawback in most of the non-invasive alternatives is the low temporal resolution accompanied by low penetration depth. A study on human brain models \cite{ErikG.Lee2016} is presented here, based on the \gls{tTIS}\footnote{Temporal refers to time-related when used in the context of electromagnetics.} method, first introduced by Grossman et al. \cite{Grossman2017} and later studied on a human brain model by Rampersad et al. \cite{Rampersad2019}. The remarkable benefits of \gls{tTIS} are the high focality of the electric field pattern and the high penetration depth. This thesis illustrates the problems arising in applying the technique by expanding the analysis of \paper{Grossman}{Grossman2017} and \paper{Rampersad}{Rampersad2019} on a population of human brain models. For this objective, an open-source Python framework \cite{thesis_repo} has been developed. The results indicate a potential correlation between the model anatomy and the final pattern, with the skull thickness being one of the major factors. Further studies need to be conducted, considering more parameters in the model anatomy impact assessment.
\vspace*{\fill}
