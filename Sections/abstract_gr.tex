\thispagestyle{plain}
\vspace*{\fill}
\begin{center}
    \LARGE
    \textit{\textbf{Περίληψη}}
        
    \vspace{0.4cm}
    \large
    \textbf{Μη επεμβατική, εν τω βάθει διέγερση του εγκεφάλου μέσω συμβαλλόντων ηλεκτρικών πεδίων}
        
    \vspace{0.4cm}
    Δημήτριος Στούπης
\end{center}
\normalsize

\vspace{0.9cm}
Η θεραπεία ή μείωση των συμπτωμάτων διαφόρων εγκεφαλοπαθειών επιτυγχάνεται αρκετά δύσκολα. Αρκετές από τις γνωστές εγκεφαλοπάθειες επηρεάζουν περιοχές που βρίσκονται σε μεγάλο βάθος μέσα στον εγκέφαλο. Υπάρχουν διάφορες επεμβατικές μέθοδοι, οι οποίες όμως εγκυμονούν μεγάλο κίνδυνο για την συνολική υγεία του ασθενούς. Αυτός είναι ένας από τους κύριους λόγους της μεγάλης ερευνητικής δραστηριότητας γύρω από τις μη-επεμβατικές μεθόδους. Δύο από τα σημαντικότερα μειονεκτήματα των μη-επεμβατικών μεθόδων είναι ή έλλειψη ακρίβειας στην στόχευση μιας περιοχής και η αδυναμία διέγερσης των εν τω βάθει περιοχών. Σε αυτή την εργασία παρουσιάζεται η μελέτη σε μοντέλα ανθρώπινων κρανίων, βασισμένη στην μέθοδο της Διακρανιακής Διέγερσης Χρονικής Συμβολής \gls{tTIS} που παρουσίασαν πρώτοι οι \paper{Grossman}{Grossman2017} και στην συνέχεια εφάρμοσαν σε μοντέλα ανθρώπινων εγκεφάλων οι \paper{Rampersad}{Rampersad2019}. Το σημαντικό πλεονέκτημα αυτής της μεθόδου είναι η δυνατότητα να φτάνει βαθιά στον εγκέφαλο, με μεγάλη χωρική ακρίβεια. Συγκεκριμένα, γίνεται συζήτηση των προβλημάτων που προκύπτουν από την αναγωγή της μελέτης των \paper{Grossman}{Grossman2017} και \paper{Rampersad}{Rampersad2019} σε έναν πληθυσμό ανθρώπινων μοντέλων. Για τον σκοπό αυτό, αναπτύχθηκε επίσης ένα λογισμικό ανοιχτού κώδικα σε γλώσσα Python \cite{thesis_repo}. Τα αποτελέσματα δείχνουν ότι υπάρχει συσχέτιση μεταξύ της ανατομίας του εκάστοτε μοντέλου με την τελική κατανομή του ηλεκτρικού πεδίου στον εγκέφαλο. Παρόλα αυτά, απαιτείται εκτενέστερη έρευνα και ανάλυση, λαμβάνοντας υπόψιν περισσότερες παραμέτρους που ενδεχομένως επηρεάζουν τα αποτελέσματα.
\vspace*{\fill}
