\pagebreak
\chapter{Discussion \& Conclusions}
\label{sec:discussion}

The revolutionary idea of Grossman and his team sparked a cascade of research for potential translation of the \gls{tTIS} method to humans. \paper{Dmochowski}{Dmochowski2017} summarizes nicely some potential uses of the method, while \paper{Esmaeilpour}{Esmaeilpour2019} provides an insight on potential targets that this method can modulate deep inside the brain. The main reason behind all this activity is that \gls{DBS} can be impactful for patients with Parkinson's disease \cite{Cagnan2019}, depression, stroke \cite{Bao2020}, and many more \cite{Cagnan2019,Polania2018} pathological conditions since the modulation of specific areas can alleviate or treat many symptoms.

Aiding in this effort to assess the impact that \gls{tTIS} might have, this work investigates the effect of head anatomy and electrode position variation through a software tool developed for this purpose. During the development of the software, it became evident that the use of it expands to more domains than Neuroscience or the intended application for this work. The software is based on the SfePy \cite{Cimrman2019}, PyMesh \cite{pymesh} and TetGen \cite{tetgen} libraries to execute the required tasks for the \gls{FEM} solution. It was build with ease of use in mind and that makes it easily scalable to other domains of engineering and science. The only thing that needs to be parsed in the solution software is the meshed model and the relevant settings, while small modifications in the relevant classes can transform the tool from a \gls{tTIS} studying framework, to a structural analysis one. Apart from the modularity, an important point is that all the code base in open-source and can be found in this work's repository \cite{thesis_repo}; thus, community engagement and potential improvement is possible.

Using the software mentioned above and running various simulations on a variety of models from the \gls{PHM} repository \cite{Lee2016,Lee2018,ErikG.Lee2016}, it became evident that the anatomical differences between individuals have a significant impact on the effectiveness of the \gls{tTIS} method. The thickness of the skin, skull and the \gls{CSF} combined plays a \draft{significant} role in the final pattern of the electric field deep inside the brain. The grater the thickness is, the more diffuse the pattern is in the white matter region \textit{(where most of the regions of interest lie)}. Furthermore, the electrode pair positioning affects the end pattern and depending on which positioning system is used, greater granularity can be achieved. For example the 10-10 system provides more flexibility on choosing the combinations of the electrode pairs over the 10-20 system; thus, the pattern is more "steerable".

On all simulations, all areas were considered uniform, which is not the case in reality. The skull for example contains the spongiform bone and the white matter density varies depending on the location. Specifically at the traditional \gls{DBS} regions \textit{(globus palidus, pineal gland)}, the white matter bundles have different conductivities and varying densities. That said to gain a more precise insight on the effectiveness of the \gls{tTIS} method, all these non-uniformities have to be taken into account, either by approximating or accurately representing them on the simulated models. The work from \paper{Rampersad}{Rampersad2013_skull_approximations} showcases how much the skull uniformity impacts the \gls{tDCS} simulations, if the layered structure of the human skull is not taken into account. The results from \cite{Rampersad2013_skull_approximations} show a significant impact on the electric field inside the brain; therefore, it is deemed necessary to be included in the simulations, if more accurate representation is required.

Combining this work's findings about the effect of the head's anatomy and the electrode positioning, with those of \paper{Rampersad}{Rampersad2019} about the effect of current ratio between the electrode pairs, one can see that there is great variability between the different head geometries. This observation leads to the conclusion for the need of optimizing the electrode placement and the current ratio, to achieve the desired effect on the different human models. The personalized optimization is deemed necessary if the same effect is to be achieved across the population and potentially achieve the benefits that this method can offer. Although it is seemingly simple to optimize, this is not true since the problem is non-convex and non-linear, making it difficult to find an optimal solution for each individual. The major problem, apart from the pure mathematical issues, is the convergence time, since a time in the order of days is impractical for most applications. The currently discussed observations, open up room for further investigation of the personalized optimization to achieve targeted, deep brain stimulation.

Lastly, the major leap will be translating all the simulations and modelling into clinical trials to assess the method's effectiveness in the real environment. This step will be a turning point for the success or the failure of this method, but only then there will be objective evidence of the actual effectiveness. The outcome of this work and the thoughts presented above, can be considered just another brick on the wall being built by the active research for this method.
