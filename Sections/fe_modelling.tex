\pagebreak
\chapter{Finite Element Modelling}

All models are discretized with a \gls{FE} mesh, before being passed to the solver. Although there are numerous \gls{FEM} solver tools, \textit{SfePy} is used in this work \cite{Cimrman2019}, since it provides all the necessary means for solving the problem in question and it has direct interface to Python. In this section, a detailed overview of the models used (\autoref{sec:fem_models}), the meshing strategy (\autoref{sec:fem_meshing}) and the solver configuration (\autoref{sec:fem_solver}) is given.

\section{Models}
\label{sec:fem_models}

A tricky part in simulations involving the brain is choosing the right or most representative model for the desired work. While spherical models are only used for proof of concept, very complex models on the other hand, increase computation time significantly. This section describes the generation of the spherical model as well as the use of publicly available realistic human head models.

\subsection{Simple Models}
\begin{wrapfigure}{r}{0.35\textwidth}
    \centering
    \includegraphics[width = 0.34\textwidth]{assets/images/sphere_brain.pdf}
    \caption{Layers of the spherical model}
    \vspace{-4cm}
    \label{fig:sphere_brain}
\end{wrapfigure}

A good approximation of the layered structure of the human head is to use the spherical model, which is divided into a total of four layers with different widths. These four layers, as seen in \autoref{fig:sphere_brain}, are the same as described in \paper{Grossman}{Grossman2017}. To generate the spherical model Gmsh \cite{gmsh}, a 3D modeling software, was used with the layer radius values as shown in \autoref{tab:spherical_layers}. The script to generate these models can be seen in \autoref{lst:sphere_code} and can also be found in the GitLab repository \cite{thesis_repo} as a snippet.
\begin{table}[!ht]
\begin{minipage}{.62\linewidth}
\centering
\caption{Layer radius of the sphere model. Base radius is considered $87\;\si{mm}$.}
\label{tab:spherical_layers}
\begin{tabular}{|c|c|c|}
\hline
\rowcolor[HTML]{C0C0C0} 
{\color[HTML]{000000} \textbf{Layer}} & {\color[HTML]{000000} \textbf{Radius {[}mm{]}}} \\ \hline
Skin & 85.96 \\ \hline
Skull & 81.61 \\ \hline
CSF & 74.21 \\ \hline
Brain & 72.21 \\ \hline
\end{tabular}
\end{minipage}
\end{table}

\noindent The radius of each layer \textit{(lower bound)} comes from the normalized layer thickness given by Grossman et al.\cite[Figure S2, J and K]{Grossman2017}, using the surface of the skin as the outer radius (base radius).

\begin{lstlisting}[language=C,caption={Sphere \gls{CAD} model generation code in \texttt{geo} format},captionpos=b, label=lst:sphere_code]
SetFactory("OpenCASCADE");

head_radius = 87;

// Ratio values from Grossman et al. 2017
brain_ratio = 0.83;
csf_ratio = 0.023;
skull_ratio = 0.085;
scalp_ratio = 0.05;

Sphere(1) = {0, 0, 0, head_radius};
Sphere(2) = {0, 0, 0, head_radius*(brain_ratio + csf_ratio + skull_ratio + scalp_ratio)};
Sphere(3) = {0, 0, 0, head_radius*(brain_ratio + csf_ratio + skull_ratio)};
Sphere(4) = {0, 0, 0, head_radius*(brain_ratio + csf_ratio)};
Sphere(5) = {0, 0, 0, head_radius*(brain_ratio)};

Physical Volume("Outer Boundary", 1) = {1};
Physical Volume("Skin", 2) = {2};
Physical Volume("Skull", 3) = {3};
Physical Volume("CSF", 4) = {4};
Physical Volume("Brain", 5) = {5};

Physical Surface("Outer Boundary", 1) = {1};
Physical Surface("Skin", 2) = {2};
Physical Surface("Skull", 3) = {3};
Physical Surface("CSF", 4) = {4};
Physical Surface("Brain", 5) = {5};

Mesh 2;
RefineMesh;
RefineMesh;
RefineMesh;

Mesh 3;
RefineMesh;
\end{lstlisting}

\subsection{Realistic Human Head Models}
\label{sec:phm_models}

Realistic brain models are difficult to construct in a \gls{CAD} software and be anatomically correct at the same time, so they are usually generated using medical imaging data from \gls{MRI} and \gls{CT} scans. The models used in the simulations of this work are from the \gls{PHM} repository of \gls{IT'IS} \cite{ErikG.Lee2016,Lee2018,ITstissue}. Each model contains a total of 7 (seven) triangular 3D surface meshes \textit{(skin, skull, \gls{CSF}, grey matter, white matter, cerebellum and ventricles)} in \gls{STL} format, generated from \gls{MRI} data. Sometimes the surface meshes are referred to as \gls{BE} meshes, and hereafter this term will be used interchangeably with \gls{STL} format.

There are also other ways, not included in this work, to model the brain and head, using \gls{MRI} data directly and generating the corresponding tetrahedral mesh from medical imaging segmentation. One streamlined and simple way to achieve this is to use \texttt{Brain2Mesh}, a tool developed by Tran et al.\cite{Tran2020}, that can generate tetrahedral meshes from segmented \gls{MR} images.

\section{Meshing}
\label{sec:fem_meshing}

Proper meshing is key to obtaining correct results when solving a \gls{FEM} problem, so in this section, the mesh quality criteria are described, as well as a basic mesh convergence analysis is presented. Besides the analysis, the followed meshing strategy is presented in detail, with specific code examples and the algorithms developed and employed. For all meshing operations, the tool PyMesh \cite{pymesh} is used, which contains all the necessary \gls{STL} file and geometry operations needed in this work. Specifically for volume meshing, this work uses TetGen \cite{tetgen}, a software library specifically designed for tetrahedral meshing.

Before proceeding with the detailed explanation of each process involved in the final generation of the meshed model, an overview diagram of the steps involved can be seen in \autoref{fig:mesh_modelling}.

\begin{figure}[H]
    \centering
    \includegraphics[width = \textwidth]{assets/images/fem_meshing_high_level.pdf}
    \caption{High-level description of the \gls{FEM} meshing procedure}
    \label{fig:mesh_modelling}
\end{figure}

\noindent The recognition of the individual mesh domains from the meshed model and subsequent labeling, is a key process that allows different material definitions in the solver, which is useful for later stages of the analysis. During the meshing procedure, a volumetric mesh is generated without labeling each domain, creating the need to recognize each area \textit{(domain)}. The recognition and subsequent labeling is done by partitioning the domains using the boundary surfaces and finding the enclosed tetrahedra, as described in \ref{subsec:elec_annotation}.

Finally, as shown in \autoref{fig:mesh_modelling}, the meshed model is saved with all necessary annotations in \gls{VTK} format, which is a very common and robust file format for \gls{FE} meshes.

\subsection{Loading Required Data}

The first step in the meshing pipeline is to import all the necessary files for geometry and electrode placement. The geometry files from the \gls{PHM} repository \cite{ErikG.Lee2016} are loaded in \gls{STL} format, which contains the surface boundaries of each region, as described in \ref{sec:phm_models}.

In addition to the geometry files, the electrode coordinate files are also loaded. The electrode coordinates are generated using \texttt{Mesh2EEG} \cite{Giacometti2014}, a tool developed by Thayer School of Engineering at Dartmouth \cite{mesh2eeg_web}.

\subsection{Electrode Array Generation \& Placement}
\label{subsec:elec_placement}

The term "electrode array generation" implies the creation of an electrode cluster from the coordinates of each electrode, to subsequently acquire the actual \gls{FE} meshes of each electrode. Ensuring repeatability across different heads \textit{(different geometries)} is key to having comparable results and drawing conclusions, so standard electrode positioning systems widely used in \gls{EEG} studies were also used here. In particular, the \textbf{international 10-20 system} \cite[chapter 13]{Malmivuo1995} is adopted, as it is the most common system in the clinical setting.

To understand where the \textbf{10-20} naming convention comes from, it is imperative to define the landmarks used. These are \textit{\gls{nasion}}, \textit{\gls{inion}}, \textit{left} and \textit{right \gls{preauricular}} and define the measurement start point for electrode placement on the head. The numbers in the name of the system refer to the percentage arc length of the arc formed by the \gls{nasion} and \gls{inion}, and the left and right \gls{preauricular}. More specifically, the number \textbf{10} denotes that the placement shall start at $10\%$ of the arc length, while the number \textbf{20} denotes the arc length that each electrode shall have, after the placement of the first one. There are denser systems than the 10-20, namely the 10-10 and the 10-5, which follow the same placement rules. Finally, it should be noted that the electrodes are placed circularly and to aid in the visualization of the placement, some details are shown in \autoref{fig:10_20_explanation}.

\begin{figure}[H]
    \centering
    \includegraphics[width = 0.85\textwidth]{assets/images/10-20_explanation.png}
    \caption[Landmarks of the 10-20 international \gls{EEG} electrode positioning system]{Landmarks of the 10-20 international \gls{EEG} electrode positioning system. \cite[figure 13.2, p.368]{Malmivuo1995}}
    \label{fig:10_20_explanation}
\end{figure}

As in a real scenario of electrode placement, these landmarks need to be identified in order to calculate the individual electrode position. This process is done manually by reviewing the 3D geometry of each model and identifying the 4 (four) landmarks required. Having the coordinates of each landmark in the coordinate system of the \gls{STL} file in question, the positions can be easily calculated with the help of the \texttt{Mesh2EEG} \cite{Giacometti2014} tool, which runs on \gls{MATLAB}. This tool takes as an input the 3D surface mesh in \gls{STL} format and the coordinates of the four landmarks on the model. The output is then the coordinates of each point in all three aforementioned systems, the 10-20, 10-10, and the 10-5.

Using the output coordinates from the \texttt{Mesh2EEG} \cite{Giacometti2014} tool, the final step for creating the electrodes is to create a cylindrical mesh for each electrode by using the calculated coordinates as the center point and orienting the cylinder normal to the surface of the head \textit{(skin)}.
\\\vspace{1pt}

\begin{wrapfigure}{l}{0.48\textwidth}
    \centering
    \vspace{-15pt}
    \includegraphics[width = 0.45\textwidth]{assets/images/eeg_electrodes_10-20.pdf}
    \caption[10-10 system names. The orange electrodes are used in the 10-20 system.]{10-10 system names. 10-20 system depicted in orange. \href{http://www.mariusthart.net/downloads/eeg_electrodes_10-20.svg}{Illustration} by \href{http://www.beteredingen.nl}{Marius 't Hart} licensed under \href{http://creativecommons.org/licenses/by-sa/3.0/nl/deed.en_GB}{CC BY-SA v3.0}}
    % \vspace{-0.5cm}
    \label{fig:electrodes_10-20}
\end{wrapfigure}

The creation of the electrode array is followed by the placement of electrodes on the head surface, a process that is one of the most important factors for the correct functioning of \gls{tTIS}. The positioning affects the stimulation pattern in the brain, therefore, by choosing the right electrodes, one can achieve the desired stimulation.

The spatial distribution of the electrodes is shown in \autoref{fig:electrodes_10-20}, where the positions marked in orange represent the \textbf{10-20} system electrodes. Before the entire mesh can be generated, each electrode must be aligned normal to the region of interest, with the alignment calculated using \autoref{alg:electrode_orientation} in \autoref{appndx:algorithms}. Finally, electrode placement on the surface of the head model is completed with the successful orientation of all electrodes and the subsequent surface mesh merge of the electrode array and the skin.

In order for the electrodes to be used by the solver, they must be annotated, i.e., each electrode is assigned a number \textit{(ID)}. The details of the annotation procedure are discussed in \ref{subsec:elec_annotation}.

\subsection{Mesh Generation}
\label{subsec:mesh_generation}

For each model, the tetrahedral mesh was generated using TetGen \cite{tetgen}, which provides the ability to selectively choose the parameters needed for modeling. TetGen \cite{tetgen} is a C++ program that can generate tetrahedral meshes from any 3D surface. To control the mesh quality, TetGen \cite{tetgen} uses the following parameters:
\begin{itemize}
	\item the \textit{radius-to-edge-ratio} $(q)$
	\item the maximum tetrahedron volume $(V_{max})$
\end{itemize}
where $q$ is the ratio of the circumference and the shortest edge length of each tetrahedron, while $V_{max}$ is the maximum allowable volume that an element can have. All quantities are given in the units used to construct each model. The command line arguments used are shown in \autoref{lst:tetgen_commands}. More details about the software compilation and integration with all other tools can be found in \autoref{chap:soft_arch}.

\begin{lstlisting}[language=bash,caption={Command to run TetGen},captionpos=b, label=lst:tetgen_commands]
	./tetgen -zpq1.414/0O4a30kANEFVV filename_merged.stl
\end{lstlisting}

As can be seen in \autoref{fig:mesh_comparison}, the mesh generated with $q = 1.414$ has almost twice as many elements as the mesh generated with $q = 2.0$. The difference can be seen in the green highlighted circular areas in \autoref{fig:mesh_comparison}, where it is prevalent that the mesh captures better the boundary surface and the number of elements in this area is greater with $q = 1.414$ than in the $q = 2.0$ case. The $q$ and $V_{max}$ presented here for the dense ($q=1.414$) mesh, are inspired by Tran et al.\cite{Tran2020}.

\begin{figure}[H]
    \centering
    \includegraphics[width = 0.85\textwidth]{assets/images/mesh_comparison.pdf}
    \caption[Mesh quality comparison for the different $q$ parameters.]{Mesh quality comparison of the different mesh parameters. The green circles highlight an area where differences are visible, in regards to mesh quality and the number of elements.}
    % \vspace{-1.2cm}
    \label{fig:mesh_comparison}
\end{figure}

Better mesh quality always comes at the cost of computation time, both for mesh generation and subsequent solution, as well as computational resources. Meshing with too many elements can prohibit the solver from running even on average desktop computers due to the high demand on \gls{RAM}. Considering computation time and resources, a trade-off must always be made to find the best balance between the desired and the actual solution. Such an evaluation is carried out through a mesh convergence study in which the problem is solved for the same model by varying the mesh parameters in each run. Details about the mesh convergence run for this work can be found in \ref{sec:mesh_quality}.

Before assessing the quality of the mesh by solving the problem, there are some basic rules that must be followed for a good mesh. One of these rules is that the \gls{tetrahedron aspect ratio}, the ratio between the circumference of each tetrahedron and the shortest edge length, should be as low as possible, which means that there will be more symmetric tetrahedra. Reducing this ratio has a direct effect on another metric, the dihedral angle. Comparing the graphs in \autoref{fig:mesh_quality_graphs} for the two different $q$-values, it is clear that the mesh with the lower $q$-value is better because the \gls{tetrahedron aspect ratio} accumulates in the areas with lower value. This ratio is key as the lower the value, the more symmetrical each tetrahedron is, eliminating potentially degenerate elements in the mesh.

\begin{figure}[H]
    \centering
    \includegraphics[width = \textwidth]{assets/images/mesh_quality_graphs.pdf}
    \caption[Mesh quality graphs for each different $q$ parameter]{Mesh quality graphs for each different $q$ parameter. The mesh statistics were generated by TetGen \cite{tetgen}.}
    % \vspace{-1.2cm}
    \label{fig:mesh_quality_graphs}
\end{figure}

To complete the evaluation of the mesh quality, the study of mesh convergence is presented in \autoref{sec:mesh_quality}, where the meshing values are chosen for this work.

\subsection{Region \& Electrode Annotation}
\label{subsec:elec_annotation}

The final step before the mesh elements are ready to be imported by the solver is to label the mesh elements. This process is straightforward when using the switch \texttt{-A} in TetGen \cite{tetgen}, which automatically assigns region attributes based on the lower boundary of each region. For the realistic head models, the assigned region attributes can be seen in \autoref{tab:domain_annotation}.

\begin{table}[!ht]
	\centering
	\caption{Mesh region labels for each mesh entity in the realistic human head models}
	\label{tab:domain_annotation}
	\begin{tabular}{|c|c|}
		\hline
		\rowcolor[HTML]{C0C0C0} 
		{\color[HTML]{000000} \textbf{Layer}} & {\color[HTML]{000000} \textbf{Number}} \\ \hline
		White Matter & 1 \\ \hline
		Grey Matter & 2 \\ \hline
		\gls{CSF} & 3 \\ \hline
		Skull & 4 \\ \hline
		Skin & 5 \\ \hline
		Cerebellum & 6 \\ \hline
		Ventricles & 7 \\ \hline
	\end{tabular}
\end{table}

As for the electrode labels, the count starts at \textbf{10} \textit{(assigned to Fp1)} and the direction of the labels is from \gls{nasion} to \gls{inion} and from left to right, with the directions as shown in \autoref{fig:electrodes_10-20}.

\subsection{Viewing the Mesh}

Visual evaluation of the mesh quality and visualization of the solution can be done using Paraview \cite{paraview} which supports the \gls{VTK} file format used in this work for all meshed models.

\section{Solver}
\label{sec:fem_solver}

After successfully meshing the model, the next step is to numerically solve the problem using a \gls{FEM}-solver. The chosen package, which contains a wealth of solvers, is SfePy \cite{Cimrman2019}. In the following sections, an overview of the boundary conditions used as well as the different solvers tested is given.

The path to the solution is a multi-step process and the steps are shown in \autoref{fig:solver_high_level}. The mesh loaded here is the one generated by the procedure described in \ref{subsec:mesh_generation}.

\begin{figure}[H]
    \centering
    \includegraphics[width = \textwidth]{assets/images/fem_solver_high_level.pdf}
    \caption{High-level description of the \gls{FEM} solver procedure}
    \label{fig:solver_high_level}
\end{figure}

\subsection{Loading the Mesh \& Settings}

There are two important files that need to be loaded into the solver. One is the meshed model in \gls{VTK} format and the other is the settings for the simulation in \gls{YAML} format. The settings file contains the path of the meshed model, the path for the various scripts, and the region labels for both the model regions (e.g. brain, skin), as well as the electrode regions, with the corresponding conductivity values per region.

It should be noted that when using SfePy's built-in mesh importer, the region labels array in the meshed model file must be named as \texttt{mat\_id}, otherwise the labels will not be imported. For the purposes of this work, this has been abstracted and further implementation details can be found in \autoref{chap:soft_arch}.

\subsection{Meshing Impact}
\label{sec:mesh_quality}

As discussed in \ref{subsec:mesh_generation}, better mesh quality helps to outline detailed shapes. Not only is the outline important, but also the actual field distribution in the solution. From \autoref{fig:solved_mesh_comparison}, it can be seen that a finer mesh leads to a better result in solving the problem. The main question is whether this difference is significant or not.

\begin{figure}[H]
    \centering
    \includegraphics[width = 0.85\textwidth]{assets/images/solved_mesh_comp.pdf}
    \caption[Problem solution comparison for the different $q$ mesh parameters.]{Problem solution comparison of the different mesh parameters. The green circles highlight an area where differences are visible, in regards to mesh outline and the number of elements.}
    \label{fig:solved_mesh_comparison}
\end{figure}

In the case of the presented $q$-values in \autoref{fig:solved_mesh_comparison}, the difference between the field distribution is prominent. The mesh with a lower $q$ value, which means that more tetrahedra are closer to being regular, has a smoother field distribution, therefore this difference can be considered significant compared to the mesh with $q=2.0$. One way to find a suitable value is to look at the changes between different runs of a solved model with varying $q$ values. In \autoref{fig:mesh_conv_study} a solved model with different mesh densities can be found.

\begin{figure}[H]
    \centering
    \includegraphics[width = 0.9\textwidth]{assets/images/mesh_conv_study.pdf}
    \caption{Meshing for different $q$ values}
    \label{fig:mesh_conv_study}
\end{figure}

Progressing from $q=2.0$ to $q=1.0$ with a step size of $0.1$, one can see in \autoref{fig:mesh_conv_study} that up to $q=1.6$ there is a difference between the patterns of the solved model, while below $q=1.4$ there is no visible difference in the field distribution. To find the appropriate value, the next step is to repeat the solution process with varying $q$ values, but now with a finer step in the range that has been identified as "converged" \textit{(no significant change from one value to the next)}. The final range can be tested on at least one more model and if there are no significant difference in the values obtained from the solver, then the mesh can be considered convergent and the final $q$ value is chosen. However, for simplicity, the value $\boldsymbol{q=1.5}$ is used here for all model meshes Finally, a more illustrative graph is provide in \autoref{fig:mesh_conv_graph} below.
\begin{figure}[H]
    \centering
    \includegraphics[width = \textwidth]{assets/images/x_axis_line_mesh_conv.pdf}
    \caption{Meshing quality graph based on \autoref{fig:mesh_conv_study}, plotted over the x-axis middle line. The values after the dash indicate the standard deviation.}
    \label{fig:mesh_conv_graph}
\end{figure}

\subsection{Solver Set-up \& Selection}

One of the most important factors in successfully solving the model is using the correct solver for the problem. Depending on the mathematical nature of the physical problem, different solvers can provide more accurate results and in less time.

In general, there are two main categories of solvers, direct and iterative, which are explained below. In this work, an iterative solver using the specific algorithms presented in \ref{subsec:iterative_solvers} is the most suitable because, as described in \ref{sec:e_ohmic_qs}, this involves solving the Laplace's equation, a second order partial differential equation.

\subsubsection{Direct Solution}

The most accurate solution to a \gls{FEM} problem is evaluated through a direct solver. This type of solver attempts to approach the solution by using traditional linear algebra techniques, such as inverse matrix computation, to find the matrix of unknowns. Direct solvers can take up huge amounts of \gls{RAM} which makes them unsuitable for a problem with a few million elements, since access to the entire matrix is required. In realistic brain models, the number of elements in a square matrix can sometimes reach the order of 6 million. Apart from the \gls{RAM} problem, these solvers are not suitable because the \gls{FE} matrices are usually very sparse, i.e., most of their elements are zero, resulting in many unnecessary zero operations.

It should be obvious that direct solvers are out of the question for realistic brain models, since they can only be used on a grid of computers. Just to compare performance, a direct solver called \gls{UMFPACK} \cite{Davis2004_umfpack} was used to solve model \texttt{126325} from the \gls{PHM} repository \cite{ErikG.Lee2016}. The results are summarized below in \autoref{tab:direct_solver_bench}. All direct solver benchmarks were obtained by running the code on the \gls{HPC} grid of \gls{AUTh} \gls{IT} center.

\begin{table}[!ht]
	\centering
	\caption{\gls{UMFPACK} direct solver benchmark using the \texttt{126325} \gls{PHM} model}
	\label{tab:direct_solver_bench}
	\begin{tabular}{|c|c|}
		\hline
		\rowcolor[HTML]{C0C0C0} 
		\textbf{Parameter} & \textbf{Value} \\ \hline
		Matrix Size & $2260222\times 2260222$ \\ \hline
		Sparsity (Fill) & $6.63\cdot 10^{-6}\%$ \\ \hline
		Execution Time & 30 min \\ \hline
		RAM Usage & 125 GB \\ \hline
	\end{tabular}
\end{table}

\subsubsection{Iterative Solution}
\label{subsec:iterative_solvers}

The logic behind this type of solvers is that initially a guess is made for the solution and then iterations follow to find the exact solution based on the given convergence criteria.

In this work, iterative solvers have been used, in particular the \gls{CG} solver. Before solving the actual problem, the matrix is first passed through a preconditioner which transforms the matrix into a more suitable form for numerical solution. Here the \gls{HYPRE} \cite{hypre-web-page} suite was used with the Boomer\gls{AMG} \cite[chapter 4]{McCormick1987_amg} preconditioner. The multigrid method is used in solving partial differential equations by hierarchical discretization of the problem. Specifically, Algebraic Multigrid constructs the hierarchy from the system matrix instead of using the differential equations directly. 

In addition, to make the system matrix less element-heavy, a coarsening algorithm is used together with the preconditioner, namely \gls{HMIS}, which in turn belongs to the \gls{HYPRE} family of tools. All these packages are included in the Python version of the \gls{PETSc} \cite{petsc-web-page,petsc-user-ref,petsc-efficient}, called \texttt{petsc4py} \cite{Dalcin2011}. \gls{PETSc} is a toolset that can be used to parallelize the problems and contains many solvers for sparse matrices \textit{(matrices with most elements zero)}, as this is the case for our problem.

\subsection{Boundary Conditions}
\label{subsec:solver_boundary_conditions}

In all simulations, the von Neumann boundary condition \textit{(flux through a surface)} $\big(\sigma\nabla\phi\cdot\vec{n} = 0\big)$ is applied on the whole model domain and the Dirichlet boundary condition \textit{(potential on a surface)} $\big(\phi = \phi_0$, $\phi = -\phi_0\big)$ is used for the active electrode and the non-active electrode, respectively. The magnitude of $\phi_0$ is determined by the desired injected current.

\subsection{Post-processing Steps}

To obtain the desired results at the end of the problem solution, there is a need to further work on the solved equations.

By solving the Laplace equation, the potential at each node is obtained and the derivative to calculate the electric field as:
\begin{equation}
	\vec{E} = -\nabla\phi
\end{equation}
where $\phi$ is the potential as solved by the \gls{FEM} solver. The combination of two different solutions with different electrode pairs is what is of interest and after calculating the potentials for each case, the electric field of the maximum, as well as the directional modulation envelope, is calculated using \autoref{alg:max_modulation_amplitude} found in \autoref{appndx:algorithms}.
