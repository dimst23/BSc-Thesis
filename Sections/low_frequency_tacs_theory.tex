\pagebreak
\chapter{Transcranial Temporal Interference Stimulation (tTIS) Theory}

To better understand the physics behind the simulations, a brief explanation regarding the conduction within the media is presented in \autoref{sec:e_ohmic_qs}, followed by an analysis regarding the \gls{tTIS} hypothesis based on Grossman et al.\cite{Grossman2017} work.

\section{Electric Field Ohmic Quasi-static Approximation}
\label{sec:e_ohmic_qs}

Generally Maxwell's equations for electromagnetic wave propagation in a medium are as follows:
% Maxwells equations
\begin{center}
\begin{minipage}{.35\linewidth}
    \begin{equation}
        \nabla\cdot\vec{E}=\dfrac{\rho}{\epsilon}
    \end{equation}
\end{minipage}
\begin{minipage}{.35\linewidth}
    \begin{equation}
        \nabla\cdot\vec{B} = 0
    \end{equation}
\end{minipage}\break
\begin{minipage}{.35\linewidth}
    \begin{equation}
        \label{eq:maxwell_curl_e}
        \nabla\times\vec{E}=-\dfrac{\partial\vec{B}}{\partial t}
    \end{equation}
\end{minipage}
\begin{minipage}{.35\linewidth}
    \begin{equation}
        \nabla\times\vec{B} = \mu\Bigg(\vec{J} + \epsilon\dfrac{\partial\vec{E}}{\partial t}\Bigg)
    \end{equation}
\end{minipage}
\end{center}

\noindent The problem in question, finding the electrical field distribution in a volume using low frequencies, can be approached by simplifying the general form of Maxwell's equations and deriving the Quasi-static approximation format. The first step is to define the assumptions taken in order for such an approach to be valid.

For the frequencies used in the studied problem, in \si{kHz} range, the displacement current can be neglected making the Ohmic currents dominate. Also, since the magnetic field is not time variant, based on \autoref{eq:maxwell_curl_e} we can write:
\begin{equation}
    \label{eq:curl_zero_e_field}
    \nabla\times\vec{E} = \vec{0}
\end{equation}
and as we know when a field is irrotational then it can be calculated from a scalar potential ($\phi$) as seen below:
\begin{equation}
    \label{eq:e_field_from_potential}
    \boxed{\vec{E} = -\nabla\phi}
\end{equation}

\noindent Moreover, since the sum of currents entering and exiting the volume is zero \textit{(Kirchhoff's second law)} we can denote:
\begin{equation}
    \nabla\cdot\vec{J} = 0
\end{equation}
where here $\vec{J}$ is the ohmic current seen below as it is assumed that there are no current sources in the volume:
\begin{equation}
    \label{eq:sigma_e_0}
    \vec{J} = \sigma\vec{E}\Rightarrow\boxed{\nabla\cdot\big(\sigma\vec{E}\big) = 0}
\end{equation}
with $\sigma$ being the electrical conductivity of each medium. Finally, based on \cref{eq:e_field_from_potential,eq:sigma_e_0} the final relationship describing the problem can be derived:
\begin{equation}
    \label{eq:laplace_e}
    \boxed{\nabla\cdot(\sigma\nabla\phi) = 0}
\end{equation}

\autoref{eq:laplace_e} describes the problem of conduction, using bulk conductors, but it is worth noting that the equation is only valid in the frequency domain where the displacement current effects are negligible and only when there is no charge generated \textit{(charge is conserved)}. Furthermore, care shall be taken with the conductivity values, since depending on the material it may be dependant on the frequency used for the stimulation.

\pagebreak
\section{Temporal Interference}
The utilization of \gls{tTIS} to achieve targeted \gls{DBS} was first introduced by Grossman et al.\cite{Grossman2017}. This technique takes advantage of the spatial electromagnetic wave interference, using frequencies at the \si{kHz} range, having almost no effect on neurons since it is known that they do not respond to higher than 1\si{kHz} \draft{(citation needed)} frequencies.
\\\vspace{1pt}

\begin{wrapfigure}{r}{0.48\textwidth}
    \vspace{-10pt}
    \centering
    \includegraphics[width = 0.44\textwidth]{assets/images/brain_figure_ttis.pdf}
    \caption[Depiction of the \gls{tTIS} pattern and the vector direction of the electric field. The purple area is the \gls{ROI} where interference happens.]{Depiction of the \gls{tTIS} pattern and the vector direction of the electric field. The purple area is the \gls{ROI} where interference happens. Image by \href{https://pixabay.com/users/openclipart-vectors-30363/?utm_source=link-attribution&amp;utm_medium=referral&amp;utm_campaign=image&amp;utm_content=150935}{OpenClipart-Vectors} from \href{https://pixabay.com/?utm_source=link-attribution&amp;utm_medium=referral&amp;utm_campaign=image&amp;utm_content=150935}{Pixabay}}
    \label{fig:brain_elec_demo}
\end{wrapfigure}

Using two pairs of electrodes (\autoref{fig:brain_elec_demo}), with each pair having a slightly different frequency, an interference pattern can be generated in the conducting medium oscillating at the difference of the two frequencies. Depending on the nature of the medium the pattern can vary and as illustrated in Grossman et al.\cite{Grossman2017} when a uniform medium is used, it is simple and easy to calculate.

Since the modulation happens in the 3D space, there will be different patterns in the $x$, $y$ and $z$ directions. According to Grossman et al.\cite[page 20]{Grossman2017}, at any location $\vec{r} = (x,y,z)$ the envelope amplitude of the \gls{AM} of the electric field produced by the temporal interference, it is calculated as:
\begin{equation}
    \label{eq:directional_amplitude}
    \vec{E}(\vec{n},\vec{r}) = \Big|\big|(\vec{E_1} + \vec{E_2})\cdot\vec{n}\big| - \big|(\vec{E_1} - \vec{E_2})\cdot\vec{n}\big|\Big|
\end{equation}
where $\vec{E_1} = \vec{E_1}(\vec{r})$, $\vec{E_2} = \vec{E_2}(\vec{r})$ are the electric fields coming from the two electrodes and $\vec{n} = \vec{n}(\vec{r})$ is the unit vector at the direction of interest.
\\\vspace{1pt}

What is of interest is the maximum amplitude of modulation at a specific location since the modulation will vary with time between zero and maximum. To calculate that amplitude across all directions, the analysis on Grossman et al.\cite[page 20]{Grossman2017} can be better supported by the analysis done on Rampersad et al.\cite[section 2.5]{Rampersad2019} and based on the two aforementioned publications, a complete description will be given here. The formula to calculate the maximum modulation amplitude, along all directions at a specific location, $\vec{r} = (x,y,z)$, is:
\begin{equation}
    \label{eq:max_mod_amplitude}
    \vec{E}_{AM}^{max}(\vec{r}) = \begin{cases}
        2\big|\vec{E_2}\big| & \text{if}\; \big|\vec{E_2}\big| < \big|\vec{E_1}\big|\cos\alpha \\
        &\\
      2\dfrac{\Big|\big|\vec{E_2}\big|\times\big(\vec{E_1} - \vec{E_2}\big)\Big|}{\big|\vec{E_1} - \vec{E_2}\big|} & \text{otherwise}
    \end{cases}
\end{equation}
where $\alpha$ is the angle between $\vec{E_1}$ and $\vec{E_2}$, while \autoref{eq:max_mod_amplitude} holds true only if $\alpha < 90\si{\degree}$. Whenever $\alpha \geq 90\si{\degree}$, the sign of one of the two fields can be flipped \textit{(it must be done is a consistent fashion)} since reaching peak field strength at different time points across different areas, is what makes the $< 90\si{\degree}$ rule to be violated. Such a change is possible considering that \autoref{eq:max_mod_amplitude} calculates the maximum effect over one oscillation, so the overall effect is the one that we care. The calculation of $\vec{E}_{AM}^{max}$ can be seen in \autoref{alg:max_modulation_amplitude} at \autoref{appndx:algorithms}.
