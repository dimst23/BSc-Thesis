\pagebreak
\chapter{Simulation Results}

The entirety of this work is based on the method introduced by \paper{Grossman}{Grossman2017}, which has been tested on murine brains and the results where very close with the simulated ones. Although on murine brain the method can produce the desired results, on the human brain it is much more difficult to achieve the required outcome and much more difficult to accurately simulate it.

\paper{Grossman}{Grossman2017} simulation models were based on a simple spherical layered model, with uniform density of the material. Such an approach is working for small brains, like murine brain, but it fails for human models where the structure is very complex and greatly varying between subjects. One way to better approach the problem on humans is to use detailed models, capturing the non-uniformities and geometric details, giving results closer to reality.

The scope of this work is to try and investigate the potential use of the \paper{Grossman}{Grossman2017} method on humans, while noting the road blocks of accurate simulation. Furthermore, since the method's effectiveness is heavily based on the injected current ratio and the electrode pair positioning, a discussion is provided in \autoref{sec:discussion} for potential optimization utilization. In the current section \paper{Grossman}{Grossman2017} approach with the spherical model is used as a benchmark for the validation of the software developed and the method is further extended for human models, where the results are seen in \autoref{sec:realistic_human_models}.

\section{Simple Spherical Layered Model}

The simple model is comprized of 4 layers in total, with varying thickness each. The layer thickness is included in \autoref{tab:spherical_layers} as a radius fraction and the values for each layer come from Grossman et al. \cite[Figure S2, J and K]{Grossman2017} \textit{(referred hereafter as paper)}. Additionally, the conductivity values used in the paper are summarized in \autoref{tab:grossman_conductivity_vals}.

\begin{table}[!ht]
\centering
\caption{Conductivity values for the simple model taken from \cite{ITstissue}}
\label{tab:grossman_conductivity_vals}
\begin{tabular}{|c|c|c|}
    \hline
    \rowcolor[HTML]{C0C0C0} 
    {\color[HTML]{000000} \textbf{Layer}} & {\color[HTML]{000000} \textbf{Conductivity {[}S/m{]}}} \\ \hline
    Skin & 0.17 \\ \hline
    Skull & 0.003504 \\ \hline
    CSF & 1.776 \\ \hline
    Brain & 0.234 \\ \hline
\end{tabular}
\end{table}

To benchmark this work's software the same configurations as in the paper were used and the results can be seen in \autoref{fig:grossman_thesis_comparison}.

\begin{figure}[H]
    \centering
    \begin{subfigure}[b]{0.49\textwidth}
        \centering
        \includegraphics[width = \textwidth]{assets/images/grossman_sphere_electric_field.png}
        \caption{Figure 2B(i) from \paper{Grossman}{Grossman2017}}
        \label{fig:grossman_envelope}
    \end{subfigure}
    \begin{subfigure}[b]{0.49\textwidth}
        \centering
        \includegraphics[width = \textwidth]{assets/images/grosman_benchmark.png}
        \caption{Solution of this work's software}
        \label{fig:envelope_at_y_benchmark}
    \end{subfigure}
    \caption[Modulation envelope on the y-direction for the spherical layered model]{Modulation envelope on the y-direction for the spherical layered model, same as on Gross et al. \cite[Figure 2B]{Grossman2017}.}
    \label{fig:grossman_thesis_comparison}
\end{figure}

It is clear from the figures above that the software can be considered accurate, based on the paper's results, so the analysis can proceed on the realistic human head models in the following section.


\section{Realistic Human Models}
\label{sec:realistic_human_models}

Going from a spherical model, to a murine model and then to a realistic human model is great leap from simplicity to great complexity. To be able and test any hypothesis in clinical trials and get validated, first the processes of modelling, testing on murine models and phantoms have to precede.

As it has been explained so far, \gls{tTIS} shows great potential to be used for non-invasive \gls{DBS}, thus the leap from the simple spherical model of \paper{Grossman}{Grossman2017} to realistic human models must be made. Two key factors have been tested in this work and more specifically the results of the effect of the human head anatomy variation as well as the variation of the electrode position are presented in \cref{subsec:effects_of_model_anatomy,subsec:varying_electrode_position}. As it will be evident below, the anatomy of each model, i.e., of each person has a significant impact on the final pattern inside the brain, thus there is the need for optimization to achieve stimulation of a small area, as different electrode combinations must be used for the each person or model. The optimization is not included in this work as it is part of future work and currently it is at a preliminary stage, however a discussion is provide on \autoref{sec:discussion}.

Lastly, it should be noted that for the electrode positioning impact assessment, models with the international 10-10 system have also been tested to find the potential issues that the 10-20 coarse nature might have.

\subsection{Effects of Model Anatomy}
\label{subsec:effects_of_model_anatomy}

Assessing the impact of the model anatomy on the field distribution requires many different subjects to have some statistically significant results. Specifically in this work, 9 models from the \gls{PHM} repository \cite{ErikG.Lee2016} were used, having 10-20 as the electrode system. Electrodes P8, F7 and P7, F8 as seen in \autoref{fig:electrodes_10-20} were used for the base vcc and base gnd respectively.

\draft{
    \begin{itemize}
        \item Run a sweep analysis on the different models to get some statistics for the effect of the model anatomy
        \item Add the graph for showing the model comparison across the axis lines to defend the difference in anatomy linked to difference in the electric field distribution
        \item Write about the model assumptions, such as uniform white matter in most models, no specific areas on the white matter, no spongiform bone included
        \item Write about potential improvements from having a conductivity calibrated for spongiform bone too
        \item Write some potential approximations that can be used \cite{Rampersad2013_skull_approximations}
    \end{itemize}
}

\subsection{Varying Electrode Position}
\label{subsec:varying_electrode_position}

\draft{
    \begin{itemize}
        \item Run a same model, varying electrode position analysis to show the pattern change when the electrode pairing changes
    \end{itemize}
}

\cite{ITstissue}
%%% PHM model citations
\cite{Lee2016}
\cite{Lee2018}
\cite{ErikG.Lee2016}
%%% PHM model citations
%%% My work citations
\cite{Hsu2019}
\cite{Rampersad2019}
%%% My work citations

\subsection{10-20 vs 10-10 system}

\draft{
    \begin{itemize}
        \item Show the granularity, if any, of the 10-10 and 10-20 systems for stimulation
    \end{itemize}
}
