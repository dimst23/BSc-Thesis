\pagebreak
\chapter{Simulation Results}

The entirety of this work is based on the method introduced by \paper{Grossman}{Grossman2017}, which has been tested on murine brains and the results where very close with the simulated ones. Although on murine brain the method can produce the desired results, on the human brain it is much more difficult to achieve the required outcome and much more difficult to accurately simulate it.

\paper{Grossman}{Grossman2017} simulation models were based on a simple spherical layered model, with uniform density of the material. Such an approach is working for small brains, like murine brain, but it fails for human models where the structure is very complex and greatly varying between subjects. One way to better approach the problem on humans is to use detailed models, capturing the non-uniformities and geometric details, giving results closer to reality.

The scope of this work is to try and investigate the potential use of the \paper{Grossman}{Grossman2017} method on humans, while noting the road blocks of accurate simulation. Furthermore, since the method's effectiveness is heavily based on the injected current ratio and the electrode pair positioning, a discussion is provided in \autoref{sec:discussion} for potential optimization utilization. In the current section \paper{Grossman}{Grossman2017} approach with the spherical model is used as a benchmark for the validation of the software developed and the method is further extended for human models, where the results are seen in \autoref{sec:realistic_human_models}.

\section{Simple Spherical Layered Model}

The simple model is comprized of 4 layers in total, with varying thickness each. The layer thickness is included in \autoref{tab:spherical_layers} as a radius fraction and the values for each layer come from Grossman et al. \cite[Figure S2, J and K]{Grossman2017} \textit{(referred hereafter as paper)}. Additionally, the conductivity values used in the paper are summarized in \autoref{tab:grossman_conductivity_vals}.

\begin{table}[!ht]
\centering
\caption{Conductivity values for the simple model taken from \cite{ITstissue}}
\label{tab:grossman_conductivity_vals}
\begin{tabular}{|c|c|c|}
    \hline
    \rowcolor[HTML]{C0C0C0} 
    {\color[HTML]{000000} \textbf{Layer}} & {\color[HTML]{000000} \textbf{Conductivity {[}S/m{]}}} \\ \hline
    Skin & 0.17 \\ \hline
    Skull & 0.003504 \\ \hline
    CSF & 1.776 \\ \hline
    Brain & 0.234 \\ \hline
\end{tabular}
\end{table}

To benchmark this work's software the same configurations as in the paper were used and the results can be seen in \autoref{fig:grossman_thesis_comparison}.

\begin{figure}[H]
    \centering
    \begin{subfigure}[b]{0.49\textwidth}
        \centering
        \includegraphics[width = \textwidth]{assets/images/grossman_sphere_electric_field.png}
        \caption{Figure 2B(i) from \paper{Grossman}{Grossman2017}}
        \label{fig:grossman_envelope}
    \end{subfigure}
    \begin{subfigure}[b]{0.49\textwidth}
        \centering
        \includegraphics[width = \textwidth]{assets/images/grosman_benchmark.png}
        \caption{Solution of this work's software}
        \label{fig:envelope_at_y_benchmark}
    \end{subfigure}
    \caption[Modulation envelope on the y-direction for the spherical layered model]{Modulation envelope on the y-direction for the spherical layered model, same as on Gross et al. \cite[Figure 2B]{Grossman2017}.}
    \label{fig:grossman_thesis_comparison}
\end{figure}

It is clear from the figures above that the software can be considered accurate, based on the paper's results, so the analysis can proceed on the realistic human head models in the following section.


\section{Realistic Human Models}
\label{sec:realistic_human_models}

Going from a spherical model, to a murine model and then to a realistic human model is great leap from simplicity to great complexity. To be able and test any hypothesis in clinical trials and get validated, first the processes of modelling, testing on murine models and phantoms have to precede.

As it has been explained so far, \gls{tTIS} shows great potential to be used for non-invasive \gls{DBS}, thus the leap from the simple spherical model of \paper{Grossman}{Grossman2017} to realistic human models must be made. Two key factors have been tested in this work and more specifically the results of the effect of the human head anatomy variation as well as the variation of the electrode position are presented in \cref{subsec:effects_of_model_anatomy,subsec:varying_electrode_position}. As it will be evident below, the anatomy of each model, i.e., of each person has a significant impact on the final pattern inside the brain, thus there is the need for optimization to achieve stimulation of a small area, as different electrode combinations must be used for the each person or model. The optimization is not included in this work as it is part of future work and currently it is at a preliminary stage, however a discussion is provide on \autoref{sec:discussion}.

Lastly, it should be noted that for the electrode positioning impact assessment, models with the international 10-10 system have also been tested to find the potential issues that the 10-20 coarse nature might have. It should be noted that all conductivity values for the simulated models are taken from the \gls{IT'IS} tissue database \cite{ITstissue}.

\subsection{Effects of Model Anatomy}
\label{subsec:effects_of_model_anatomy}

Assessing the impact of the model anatomy on the field distribution requires many different subjects to have statistically significant results. Specifically in this work, 9 models from the \gls{PHM} repository \cite{ErikG.Lee2016} were used, having the 10-20 as the electrode placement system. Electrodes P8, F7 \textit{(1\si{kHz})} and P7, F8 \textit{(1.04\si{kHz})} were used, with the first electrode in each couple being the active electrode. The corresponding electrode positions are shown in \autoref{fig:electrodes_10-20}.

\begin{figure}[H]
    \centering
    \begin{subfigure}[b]{0.49\textwidth}
        \includegraphics[width = \textwidth]{assets/images/y_axis_line_brain.png}
        \caption{Slice across the x-axis.}
        \label{fig:brain_slice_for_effects_x}
    \end{subfigure}
    \begin{subfigure}[b]{0.49\textwidth}
        \includegraphics[width = \textwidth]{assets/images/center_of_bounds_line_brain.png}
        \caption{Slice across the center of bounds.}
        \label{fig:brain_slice_for_effects_cf}
    \end{subfigure}
    \caption{Slice references for \cref{fig:x_axis_effect,fig:center_of_bounds_effect}. The white line represents the line of measurement.}
    \label{fig:brain_slice_for_effects}
\end{figure}

As a first step the maximum modulation envelope electric field value was taken across the line of the x-axis for each model, with the reference plane illustrated in \autoref{fig:brain_slice_for_effects_x}. The field values for each model, along the x-axis, can be seen in \autoref{fig:x_axis_effect}, along with the standard deviation in the legend.

\begin{figure}[H]
    \centering
    \includegraphics[width = \textwidth]{assets/images/x_axis_line.pdf}
    \caption{Field distribution for each \gls{PHM} model along the x-axis. The number near each model is the standard deviation for the distribution. The curves were smoothed using a moving average with a window size of 50 samples.}
    \label{fig:x_axis_effect}
\end{figure}

It is evident from the shape of the curves in \autoref{fig:x_axis_effect} that the same pattern is followed, but there is a significant variation from one model to the next, both in the curve's shape and the standard deviation. Furthermore, another measurement was made along the center of bounds axis (\autoref{fig:brain_slice_for_effects_cf}) and as illustrated in \autoref{fig:center_of_bounds_effect}, the problems discussed for the x-axis plot are even more prevalent.

\begin{figure}[H]
    \centering
    \includegraphics[width = 0.85\textwidth]{assets/images/center_of_bounds_line.pdf}
    \caption{Field distribution for each \gls{PHM} model along each model's the center of bounds. The number near each model is the standard deviation for the distribution. The curves were smoothed using a moving average with a window size of 50 samples.}
    \label{fig:center_of_bounds_effect}
\end{figure}

Looking at \autoref{fig:center_of_bounds_effect} it is evident that for each model the variability of the electric field is not negligible. The graph indicates the values on the line that cuts the center of bounds for each model, meaning that is passes through many different layers at different heights in the head, giving a clearer picture of what happens. 

To better understand the cause of this difference and gauge which models have similar patterns, \gls{PCA} was performed with the features as summarized in \autoref{tab:pca_features}. For each model, the thickness was calculated from the coordinate points along the x-axis line (\autoref{fig:brain_slice_for_effects_x}), in the corresponding regions. Each of the maximum modulation envelope electric field was calculated in the whole volume of each designated region on \autoref{tab:pca_features}, same as the standard deviation.

\begin{table}[!ht]
	\centering
    \caption[\gls{PCA} feature values]{Feature values used in the \gls{PCA} analysis. The electric field values, denoted with $E$ are the mean for each model on the designated region. $\sigma$ is the corresponding standard deviation for each electric field value. The thickness value is the sum of the skin, skull , and \gls{CSF} thickness along the x-axis line for each model.}
    \label{tab:pca_features}
    \begin{tabular}{|c|c|c|c|c|c|c|c|}
        % \hline
        \cline{2-8}
        \multicolumn{1}{c|}{} & \cellcolor[HTML]{C0C0C0}\textbf{[mm]} & \multicolumn{6}{c|}{\cellcolor[HTML]{C0C0C0}\textbf{[mV/m]}} \\ \hline
        \rowcolor[HTML]{C0C0C0} 
        \textbf{Model ID} & \textbf{Thickness} &  $\boldsymbol{E_{CSF}}$ & $\boldsymbol{E_{GM}}$ & $\boldsymbol{E_{WM}}$ & $\boldsymbol{\sigma_{CSF}}$ & $\boldsymbol{\sigma_{GM}}$ & $\boldsymbol{\sigma_{WM}}$ \\\hline
        103414 & 25.36 &             31.70 &            41.70 &            43.35 &            11.89 &           10.84 &            9.99 \\\hline
        105014 & 36.60 &             26.43 &            34.37 &            35.35 &            10.39 &            9.51 &            8.39 \\\hline
        105115 & 37.09 &             24.30 &            32.45 &            33.33 &             8.91 &            8.57 &            7.49 \\\hline
        110411 & 39.84 &             24.67 &            33.33 &            34.17 &             9.50 &            9.36 &            8.42 \\\hline
        111716 & 38.71 &             26.08 &            35.06 &            35.91 &            10.37 &           10.35 &            9.29 \\\hline
        113619 & 37.18 &             26.16 &            35.23 &            36.64 &            10.51 &           10.34 &            9.86 \\\hline
        117122 & 37.38 &             29.17 &            39.43 &            40.84 &            11.76 &           12.16 &           11.38 \\\hline
        163129 & 39.07 &             26.77 &            35.50 &            36.67 &            10.18 &            9.98 &            8.92 \\\hline
        196750 & 39.88 &             28.76 &            38.12 &            39.40 &            10.81 &           10.29 &            9.23 \\\hline
    \end{tabular}
\end{table}

Correlating the data from \autoref{tab:pca_features} in a \gls{PCA} chart, provides an insight on similarities or the lack thereof, between the models used. As seen in \autoref{fig:model_pca}, there is some clustering of models indicating that they more correlated than the rest, based on the provided feature set. That said, it is evident that the models 111716, 113619, and 163129 are clustered together, in a weak way, and as seen on \autoref{tab:pca_features} they have very similar thickness values. Models like 103414 and 110411 are on the 2 sides of the graph and they have very different thickness values. The two latter models have also significant difference in the mean electric field values on the white matter.

\begin{figure}[H]
    \centering
    \includegraphics[width = 0.85\textwidth]{assets/images/model_pca.pdf}
    \caption{Principle Component Analysis (PCA) graph for the different models, based on the features included in \autoref{tab:pca_features}. The explained variance ratio in $\boldsymbol{96.18\%}$.}
    \label{fig:model_pca}
\end{figure}

From the analysis provided above, it seems that there is a dependance on the thickness of the model with the final electric field sparsity and distribution, considering the same electrodes are operating for each one of the models. Despite the initial evidence presented above, further investigation with more models is required to reach a conclusion. What is certain is that there is a dependance on the model geometry and the final pattern as it can also be seen in \autoref{fig:models_pattern_variation}.

\begin{figure}[H]
    \centering
    \includegraphics[width = 0.8\textwidth]{assets/images/brain_pattern_variation_models.pdf}
    \caption{Variation of the electric field pattern distribution for the different models. The scale is the same for all models.}
    \label{fig:models_pattern_variation}
\end{figure}

In all models included in this work and all simulations, a uniform white matter and skull are assumed. This is not the correct since the white matter has different regions with different densities, and similarly the skull has the spongiform regions with very low conductivity value. As shown by \paper{Rampersad}{Rampersad2013_skull_approximations}, on \gls{tES} and specifically \gls{tDCS}, the different conductivity values for the layered bone structure \textit{(spongiform and compact bone)} of the skull can have a considerable effect if not considered in the simulations. The white matter uniformity consideration on the other hand, provides no information for the different regions and it puts a road block on assessing the impact of the electric field on the regions of interest, most of them lying in the white matter region, near the ventricles \textit{(e.g., pineal gland and globus palidus)}.

\subsection{Varying Electrode Position}
\label{subsec:varying_electrode_position}

Model anatomy is one of the crucial factors that must be considered when simulating with \gls{tTIS}. Although it is not the only player, since the current ratio between the two electrodes, and the position of them, can influence the results a lot.

As shown by \paper{Rampersad}{Rampersad2019}, the current ratio has significant impact regarding the maximum potential achieved in a given volume. This dependance is well explained as shown in \autoref{fig:mod_env_amplitude_var}, due to the fact that when the two electrodes are not placed symmetrically, the pair closest to the region of interest takes over. The resulting modulation amplitude is much lower than the one seen in \autoref{fig:modulation_showcase}.

\begin{figure}[H]
    \centering
    \includegraphics[width = \textwidth]{assets/images/modulation_envelope_2-1.pdf}
    \caption{Modulation envelope and the effect of differing amplitudes between the two fields}
    \label{fig:mod_env_amplitude_var}
\end{figure}

In this work, the attention will be shifted towards the electrode position variation and its subsequent effects. One can imagine that moving the electrodes or changing the combination used can have an impact on the final pattern. It is evident from \autoref{fig:elec_position_variation} that using the combination just one step above the previous one can have significant impact on the field distribution.

\begin{figure}[H]
    \centering
    \begin{subfigure}[b]{0.49\textwidth}
        \includegraphics[width = \textwidth]{assets/images/105014_26_16.png}
        \caption{P8, F8 and P7, F7 electrode pairs.}
        \label{fig:26_16_elec_pair}
    \end{subfigure}
    \begin{subfigure}[b]{0.49\textwidth}
        \includegraphics[width = \textwidth]{assets/images/105014_25_15.png}
        \caption{P4, F4 and P3, F3 electrode pairs.}
        \label{fig:25_15_elec_pair}
    \end{subfigure}
    \caption{Electric field distribution with varying electrode position}
    \label{fig:elec_position_variation}
\end{figure}

Even if one electrode is moved and the others stay as they are, significant differences are observed \textit{(\autoref{fig:one_elec_moved_var})}. One of the main reasons making \cref{fig:26_16_elec_pair,fig:25_15_elec_pair} differ significantly is the lack of granularity the 10-20 system provides. Limited only to 19 electrodes, there is not much room for movement, hence a system with more electrodes is required and in that case th 10-10 system can be a good candidate. The reason that the 10-5 system is not considered is mainly due to the very high number of electrodes \textit{(329 in total)}.

\begin{figure}[H]
    \centering
    \begin{subfigure}[b]{0.49\textwidth}
        \includegraphics[width = \textwidth]{assets/images/105014_26_16_22_12_z-axis.png}
        \caption{P8, F8 and P7, F7 electrode pairs.}
        \label{fig:26_16_12_22_elec_pair}
    \end{subfigure}
    \begin{subfigure}[b]{0.49\textwidth}
        \includegraphics[width = \textwidth]{assets/images/105014_26_16_23_12_z-axis.png}
        \caption{P8, F8 and P3, F7 electrode pairs.}
        \label{fig:25_15_13_24_elec_pair}
    \end{subfigure}
    \caption{Electric field distribution with varying electrode position}
    \label{fig:one_elec_moved_var}
\end{figure}

As it will be discussed in a more detail in the \nameref{sec:discussion}, the dependance on the position, the current ratio, and the model anatomy drives the need for seeking personalized optimization for each model. The optimization will be part of future work and it is not covered in the current work.

To better illustrate the granularity mentioned above, the \textbf{103414} model was meshed with all the 10-10 system electrodes and the solved with the pairs as shown in \autoref{fig:one_elec_moved_var_10-10}.

\begin{figure}[H]
    \centering
    \begin{subfigure}[b]{0.49\textwidth}
        \includegraphics[width = \textwidth]{assets/images/103414_35_68_43_76.png}
        \caption{FT7, P7 and FT8, P8 electrode pairs.}
        \label{fig:35_68_43_76_elec_pair}
    \end{subfigure}
    \begin{subfigure}[b]{0.49\textwidth}
        \includegraphics[width = \textwidth]{assets/images/103414_35_68_43_75.png}
        \caption{FT7, P7 and FT8, P6 electrode pairs.}
        \label{fig:35_68_43_75_elec_pair}
    \end{subfigure}
    \caption{Electric field distribution with varying electrode position on the 10-10 system.}
    \label{fig:one_elec_moved_var_10-10}
\end{figure}

What is clear from the figure above is that the effect the electrode position on the electric field pattern can be manipulated with more refined moves.
