\pagebreak
\chapter{Software Architecture}
\label{chap:soft_arch}

Finding the right software and the right tools for the specific problem requirements is usually a challenging task. There are many proprietary software available providing an intuitive user interface, and the required tools for \gls{FE} analysis. Apart from the proprietary ones there are many other open-source software like \gls{SimNIBS}, and SCIRun. The challenge is to combine the weaknesses of one software with the strengths of the other.

In this work an interface-like software has been developed for the simulations of the \gls{tTIS} method. The boundaries are not strictly set for this purpose only, but the software in question is easily scalable to multiple physics problems requiring \gls{FEM} solution. As mentioned before, this work uses SfePy \cite{Cimrman2019}, PyMesh \cite{pymesh}, and TetGen \cite{tetgen} packages. These are all open-source and actively developed, with TetGen \cite{tetgen} being highly used in application involving tetrahedral meshing.

As part of experimentation and learning, custom functions annotating and recognizing the regions of each model were developed and can be found on GitLab \cite{thesis_repo}, at \href{https://gitlab.com/ttis-simulations/ttis-software/-/releases#v0.1}{v0.1 release}. 

\section{UML Class Diagram}

Better understanding of the software's structure and function requires using \gls{UML} class and use case diagrams, highly established ways of technical communication in software engineering.

\autoref{fig:uml_class_meshing} presents the class relations and definitions of the meshing procedure. The steps taken to mesh the model are illustrated in \autoref{fig:use_case_meshing}.

\begin{figure}[H]
    \centering
    \includegraphics[width = 0.85\textwidth]{assets/images/uml_class_diagram_meshing.png}
    \caption{\gls{UML} class diagram for this work's meshing classes}
    \label{fig:uml_class_meshing}
\end{figure}

The meshing procedure is a bit involved, in terms of steps and packages required, since it generates a TetGen file which is the meshed using TetGen. The \gls{FEM} solution procedure is much simpler, with most of the functionality done by the \texttt{Solver} class automatically. The only required package for the normal operation of the solver is SfePy. The \texttt{Solver} class is highly modular and can be modified to suit the needs of the problem, by just altering the equation and the boundary condition definition.

\begin{figure}[H]
    \centering
    \includegraphics[width = 0.38\textwidth]{assets/images/uml_class_diagram_fem.png}
    \caption{\gls{UML} class diagram for this work's \gls{FEM} class}
    \label{fig:uml_class_fem}
\end{figure}

\section{Use-case diagrams}

An overview of the meshing procedure is provided in \autoref{fig:use_case_meshing}. This relies on the class implementation (\autoref{fig:uml_class_meshing}) and the relevant code executing the use cases described, can be found in \autoref{appndx:code}.

\begin{figure}[H]
    \centering
    \includegraphics[width = 0.76\textwidth]{assets/images/uml_meshing_use_case_diagram.png}
    \caption{\gls{UML} use case diagram for the meshing procedure}
    \label{fig:use_case_meshing}
\end{figure}

Same as above, \autoref{fig:use_case_fem} illustrates the procedure to solve the meshed model. The code regarding these use cases can be found in \autoref{appndx:code}.

\begin{figure}[H]
    \centering
    \includegraphics[width = 0.76\textwidth]{assets/images/uml_fem_use_case_diagram.png}
    \caption{\gls{UML} use case diagram for the \gls{FEM} solution procedure}
    \label{fig:use_case_fem}
\end{figure}
