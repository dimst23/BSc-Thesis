\documentclass[titlepage, a4paper, 10pt]{report}
\usepackage[top=3.5cm, left=3cm, right=3cm, bottom=2.5cm]{geometry}
\usepackage[english]{babel}
\usepackage[unicode]{hyperref}
\usepackage{csquotes}
\usepackage{siunitx}
\usepackage{wrapfig}
\usepackage{footnote}
\usepackage[ruled,vlined,linesnumbered,longend]{algorithm2e}
\usepackage{titlesec}
\usepackage{titletoc}
\usepackage{caption}
\usepackage{subcaption}

% Citations and glossaries
\usepackage[toc,nonumberlist,nopostdot,acronyms,automake]{glossaries}
\usepackage[style=ieee]{biblatex}
\usepackage{bookmark}

% Color hyperlinks instead of having a box around them
\usepackage{listings}
\usepackage[table,xcdraw]{xcolor}
\usepackage{graphicx}
\usepackage{tikz}
\usepackage[labelfont=bf]{caption}
\usepackage{float}

% Maths packages
\usepackage{amsmath, amssymb}
\usepackage{mathtools}
\usepackage{nicefrac}
\usepackage{multicol}

\usepackage{cleveref}

% Commands
\newcommand{\paper}[2]{#1 et al.\cite{#2}}
\newcommand{\thetitle}{Investigating non-invasive deep brain stimulation using temporally interfering electric fields}

% Colors
\definecolor{dark_grey}{HTML}{303030}

% PDF properties
\hypersetup{
  pdfauthor   = {Dimitrios Stoupis},
  pdftitle    = {\thetitle},
  pdfsubject  = {Bachelor thesis at the Department of Physics of the Aristotle University of Thessaloniki},
  pdfkeywords = {Neuroscience} {tTIS} {Thesis} {Brain} {Temporal Interference},
  colorlinks  = true, %Colours links instead of ugly boxes
  urlcolor    = blue, %Colour for external hyperlinks
  linkcolor   = dark_grey, %Colour of internal links
  citecolor   = blue %Colour of citations
}

\addbibresource{references.bib}

%\title{tACS using temporal interference to achieve targeted deep brain stimulation}
\title{\thetitle}
\author{Dimitrios Stoupis}
\date{}

\makeglossaries

% Highlight draft areas
\newcommand{\draft}[1]{{\leavevmode\color{red!90!black}#1}}

% Listing style
\definecolor{codegreen}{rgb}{0,0.6,0}
\definecolor{codegray}{rgb}{0.5,0.5,0.5}
\definecolor{codepurple}{rgb}{0.58,0,0.82}
\definecolor{backcolour}{rgb}{0.95,0.95,0.92}

\lstdefinestyle{mystyle}{
    backgroundcolor=\color{backcolour},   
    commentstyle=\color{codegreen},
    keywordstyle=\color{magenta},
    numberstyle=\tiny\color{codegray},
    stringstyle=\color{codepurple},
    basicstyle=\ttfamily\footnotesize,
    breakatwhitespace=false,         
    breaklines=true,                 
    captionpos=b,                    
    keepspaces=true,                 
    numbers=left,                    
    numbersep=5pt,                  
    showspaces=false,                
    showstringspaces=false,
    showtabs=false,                  
    tabsize=2
}

\lstset{style=mystyle}
\pagenumbering{roman}

% Increase section depth
\setcounter{secnumdepth}{4}
\setcounter{tocdepth}{4}
\titleformat{\paragraph} [hang] {\normalfont\normalsize\bfseries} {\theparagraph} {1em} {}
%% End of package operations

\begin{document}

\maketitle

\begin{abstract}
  Treating brain diseases or suppressing the symptoms is not a trivial task. Different methods exists and usually these are invasive, which inherently poses a great risk for the overall health of the individual. This one of the main reasons that there is a lot of research activity in the non-invasive methods and more specifically how to electromagnetically induce potentials or alter the behavior of neuronal networks. One drawback in most of the methods is the low temporal resolution accompanied by low penetration depth. Here a study on a set of human brain models\cite{ErikG.Lee2016} is presented, based on the \gls{tTIS} method, first introduced by Grossman et al.\cite{Grossman2017} and studied on a human brain model by Rampersad et al.\cite{Rampersad2019}. The scope of this report is to illustrate the potential problems or traits that will arise by simulating on a population human brain models. \draft{Info about the results required.}
\end{abstract}
\pagebreak


% \thispagestyle{empty}
% \setcounter{page}{0}
\tableofcontents
% \clearpage

% \pagebreak
% \cleardoublepage
% \phantomsection
% \addcontentsline{toc}{section}{\listfigurename}
\listoffigures
% \cleardoublepage
% \phantomsection
% \addcontentsline{toc}{section}{\listtablename}
\listoftables
% \cleardoublepage
% \phantomsection
% \addcontentsline{toc}{section}{\lstlistlistingname}
\lstlistoflistings

% Add the sections
\pagebreak
\pagenumbering{arabic}
\chapter{Introduction}

Since neurons are driven by electric currents, changing their surrounding environment's electrical conditions can alter their behavior. Both electrically and magnetically induced stimulation has been used for many years, although the latter was studied several years earlier. Each method is based on different physics, although the results are similar.

\begin{wrapfigure}{l}{0.40\textwidth}
    \centering
    \includegraphics[width = 0.3\textwidth]{assets/images/cortical_layers.pdf}
    \caption{Cellular structure of the neocortex. (Purves et al.\cite{Purves2012}, Figure 27.1(B) p.628)}
    \label{fig:cortical_layers}
\end{wrapfigure}

Because of the layered structure (\autoref{fig:cortical_layers}) of the neocortex, the electric fields induced by \gls{TMS} act mainly on interneurons and collaterals of the tangentially oriented pyramidal cells, whereas the total electric field induced by electrical stimulation is perpendicular to the cortical surface and thus acts mainly on pyramidal cells of \textit{layer V} since most of them are perpendicular to the cortex. As a result, \gls{TMS}-induced fields have a low penetration depth. In contrast, \gls{tES} induced fields can penetrate deep into the brain, but at the expense of spatial resolution. Therefore, \gls{tES} is mostly used to try to reach deep targets. 

\gls{DBS}, particularly the non-invasive methods achieving it, is an emerging need as most crucial brain components' functionality lies deep within the brain. Although successful invasive methods can alleviate Parkinson's symptoms, non-invasive methods offer greater scalability and lower risk to the patient. However, the challenge with non-invasive methods is the lack of spatial resolution and precise targeting of the desired area, usually on the order of tens of millimeters (e.g., pituitary gland \cite{Yadav2017_pituitary}).

In 2017, a new method that seemingly revolutionizes the way non-invasive \gls{DBS} is performed by accommodating clever tricks or \textit{"hacks"} of the neuron physics, was presented by Nir Grossman and his team \cite{Grossman2017} for the first time. The method of \paper{Grossman}{Grossman2017} uses a pair of electrodes operated with a high-frequency alternating current whose envelope oscillates at the desired frequency by setting different frequencies between the pairs of electrodes. The significant advantage is reaching deep targets and stimulating only the regions of interest without affecting surrounding areas or the entire brain.

The Grossman method is a promising technique for the future of \gls{DBS}. It is investigated in this work to evaluate its potential application in humans, as only sphere and simple phantom models were studied in the original work \cite{Grossman2017}, in conjunction with successful experiments in murine models. Expanding the study on realistic human brain models will help understand \gls{tTIS} specifics, and tests on healthy individuals will consolidate the method's effectiveness. Ultimately, this work aims to enable optimized and personalized treatment planning for humans based on rigorous computer simulations.

As a closing remark, it has been a firm belief of the author that the greatest benefit to science and society comes from the open sharing of knowledge, and true to this belief, all the code, drafts, text, and all originally created materials of this work are publicly available on the GitLab repository \cite{thesis_repo}.

\section{Methods of Electrical Stimulation}

To better understand the primary advantages of the \gls{tTIS} method, the different methods for achieving electrical stimulation shall be touched upon for comparison purposes. An overview of the two main methods is presented in this section. These are the most commonly used ones and form the basis for almost all other electrical stimulation method variants.

\subsection{Transcranial Direct Current Stimulation \textit{(tDCS)}}

\gls{tDCS} delivers constant current via electrodes placed on the head, targeting to stimulate the cortical areas. Numerous studies show promising results in depression treatment \cite{Moffa2020,Brunoni2016}, and \gls{tDCS} is increasingly used in such cases \cite{Nitsche2008}.

Although \gls{tDCS} is useful for overall cortical stimulation, the lack of focality is one of the principal issues. As shown in \autoref{fig:tdcs_pattern}, the electric current flows through an extensive area, potentially affecting unwanted regions of the brain.

\begin{figure}[H]
    \centering
    \includegraphics[width = 0.75\textwidth]{assets/images/tdcs_pattern.png}
    \caption{\gls{tDCS} pattern showcase with P7 \textit{(active)}, F7 electrodes based on the 10-20 system.}
    \label{fig:tdcs_pattern}
\end{figure}

Furthermore, achieving neural firing synchronization is impossible through \gls{tDCS} since this kind of effect requires alternating current. The method that has such an ability is \gls{tACS}.

\subsection{Transcranial Alternating Current Stimulation \textit{(tACS)}}

In many aspects, \gls{tACS} and \gls{tDCS} are identical. As mentioned before, the key characteristic where \gls{tACS} differs is the neural firing synchronization. Like \gls{tDCS}, this method also lacks focality.

\gls{tACS} requires using low-frequency currents in the order of tens of Hz since at those frequencies, the neurons are firing the action potentials. The problem arising here again is the stimulation of a large area, synchronized at the stimulating current frequency. Such synchronization can have adverse effects, i.e., stimulating and synchronizing unwanted areas.

Stimulating with higher frequency currents renders the method useless because the neuronal membrane can either not follow the oscillation, or the frequency is too high for the desired synchronization. This problem is solved by the \gls{tTIS} method, which is the focus of this work.

\subsection{Transcranial Temporal Interference Stimulation \textit{(tTIS)}}

\gls{tTIS} is essentially \gls{tACS} with a small, yet remarkable, modification. Taking advantage of the superposition principle of the electric field and utilizing the emerging interference patterns is the essence of this approach. The current flowing through each electrode pair has a high enough frequency that does not modulate any neurons, as opposed to \gls{tACS}, where the frequency shall be in the target modulation range. The modulator here is the envelope (\autoref{fig:modulation_showcase}) of the two electric fields' temporal interference, given that the envelope modulates at the difference of the two electrode frequencies.

\begin{figure}[H]
    \centering
    \includegraphics[width = 0.95\textwidth]{assets/images/modulation_envelope.pdf}
    \caption{Wave superposition pattern with equal amplitude waves}
    \label{fig:modulation_showcase}
\end{figure}

This method was tested by \paper{Grossman}{Grossman2017} on murine brains, and the results were spot on with the respective simulation. Targeted neuro-modulation was achieved with a much better focality and depth than the traditional methods. Although this process works well on murine brains, it poses a great challenge for the human models as the geometry and the volume change dramatically. A sample image from this work can be seen in \autoref{fig:ttis_pattern}, where the Grossman's method focality and depth in comparison with \autoref{fig:tdcs_pattern} are visible.

\begin{figure}[H]
    \centering
    \includegraphics[width = 0.75\textwidth]{assets/images/ttis_pattern.png}
    \caption{Electric field pattern of the \gls{tTIS} method using P8 \textit{(active)}, F8 as the base frequency electrodes and P7 \textit{(active)}, F7 as the secondary \textit{(delta)} frequency electrodes. The base frequency is $f = 1\; kHz$ and $\Delta f=40\; kHz$.}
    \label{fig:ttis_pattern}
\end{figure}

This approach seems promising. It is yet to be reinforced by follow-up works and real experiments on human subjects in the clinical setting, potentially aiding in treating some diseases.

\section{Theoretical Background}

To grasp the physics behind the simulations, a brief explanation regarding current conduction within tissues is presented in \autoref{sec:e_ohmic_qs}, followed by an analysis regarding the \gls{tTIS} hypothesis based on Grossman et al.\cite{Grossman2017} work.

\subsection{Electric Field Ohmic Quasi-static Approximation}
\label{sec:e_ohmic_qs}

Generally, Maxwell's equations for electromagnetic wave propagation in a medium are as follows:
% Maxwells equations
\begin{center}
\begin{minipage}{.35\linewidth}
    \begin{equation}
        \nabla\cdot\vec{E}=\dfrac{\rho}{\epsilon}
    \end{equation}
\end{minipage}
\begin{minipage}{.35\linewidth}
    \begin{equation}
        \nabla\cdot\vec{B} = 0
    \end{equation}
\end{minipage}\break
\begin{minipage}{.35\linewidth}
    \begin{equation}
        \label{eq:maxwell_curl_e}
        \nabla\times\vec{E}=-\dfrac{\partial\vec{B}}{\partial t}
    \end{equation}
\end{minipage}
\begin{minipage}{.35\linewidth}
    \begin{equation}
        \nabla\times\vec{B} = \mu\Bigg(\vec{J} + \epsilon\dfrac{\partial\vec{E}}{\partial t}\Bigg)
    \end{equation}
\end{minipage}
\end{center}

\noindent The problem in question, finding the electrical field distribution in a volume using low frequencies, can be approached by simplifying the general form of Maxwell's equations and deriving the Quasi-static approximation format. The first step is to define the assumptions taken for such an approach to be valid.

For the frequencies in the \si{kHz} range used in the studied problem, the displacement current can be neglected, making the Ohmic currents dominant. Also, since the magnetic field is not time-variant, based on \autoref{eq:maxwell_curl_e}, we can write:
\begin{equation}
    \label{eq:curl_zero_e_field}
    \nabla\times\vec{E} = \vec{0}
\end{equation}
and as we know, when a field is irrotational, then it can be calculated from a scalar potential ($\phi$) as seen below:
\begin{equation}
    \label{eq:e_field_from_potential}
    \boxed{\vec{E} = -\nabla\phi}
\end{equation}

\noindent Moreover, since the sum of currents entering and exiting the volume is zero \textit{(Kirchhoff's second law)}, we can denote:
\begin{equation}
    \nabla\cdot\vec{J} = 0
\end{equation}
where here $\vec{J}$ is the ohmic current seen below as it is assumed that there are no current sources in the volume:
\begin{equation}
    \label{eq:sigma_e_0}
    \vec{J} = \sigma\vec{E}\Rightarrow\boxed{\nabla\cdot\big(\sigma\vec{E}\big) = 0}
\end{equation}
with $\sigma$ being the electrical conductivity of each tissue. Finally, based on \cref{eq:e_field_from_potential,eq:sigma_e_0} the final relationship describing the problem can be derived:
\begin{equation}
    \label{eq:laplace_e}
    \boxed{\nabla\cdot(\sigma\nabla\phi) = 0}
\end{equation}

\autoref{eq:laplace_e} describes the problem of conduction using bulk conductors. However, it is worth noting that the equation is only valid in the frequency domain, where the displacement current effects are negligible and only when there is no charge generated \textit{(charge is conserved)}. Furthermore, care shall be taken with the conductivity values, which may depend on the stimulating frequency for different materials.

\pagebreak
\subsection{Temporal Interference}
The utilization of \gls{tTIS} to achieve targeted \gls{DBS} was first introduced by Grossman et al.\cite{Grossman2017}. This technique takes advantage of the spatial electromagnetic wave interference, using frequencies at the \si{kHz} range, having almost no effect on neurons, which do not respond to higher than 1\si{kHz} frequencies \cite{Hutcheon2000}.
\\\vspace{1pt}

\begin{wrapfigure}{r}{0.48\textwidth}
    \vspace{-10pt}
    \centering
    \includegraphics[width = 0.44\textwidth]{assets/images/brain_figure_ttis.pdf}
    \caption[Depiction of the \gls{tTIS} pattern and the vector direction of the electric field. The purple area is the \gls{ROI} where interference happens.]{Depiction of the \gls{tTIS} pattern and the vector direction of the electric field. The purple area is the \gls{ROI} where interference happens. Image by \href{https://pixabay.com/users/openclipart-vectors-30363/?utm_source=link-attribution&amp;utm_medium=referral&amp;utm_campaign=image&amp;utm_content=150935}{OpenClipart-Vectors} from \href{https://pixabay.com/?utm_source=link-attribution&amp;utm_medium=referral&amp;utm_campaign=image&amp;utm_content=150935}{Pixabay}}
    \label{fig:brain_elec_demo}
\end{wrapfigure}

Using two pairs of electrodes (\autoref{fig:brain_elec_demo}), with each pair having a slightly different frequency, an interference pattern can be generated in the conducting medium oscillating at the two frequencies' difference. Depending on the medium's nature, the pattern can vary, and as illustrated in Grossman et al.\cite{Grossman2017}, when a uniform medium is used, the pattern easily calculated.

Since the modulation happens in the 3D space, there will be different patterns in the $x$, $y$ and $z$ directions. According to Grossman et al.\cite[page 20]{Grossman2017}, at any location $\vec{r} = (x,y,z)$ the envelope amplitude of the \gls{AM} of the electric field produced by the temporal interference is calculated as:
\begin{equation}
    \label{eq:directional_amplitude}
    \vec{E}(\vec{n},\vec{r}) = \Big|\big|(\vec{E_1} + \vec{E_2})\cdot\vec{n}\big| - \big|(\vec{E_1} - \vec{E_2})\cdot\vec{n}\big|\Big|
\end{equation}
where $\vec{E_1} = \vec{E_1}(\vec{r})$, $\vec{E_2} = \vec{E_2}(\vec{r})$ are the electric fields coming from the two electrodes and $\vec{n} = \vec{n}(\vec{r})$ is the unit vector at the direction of interest.
\\\vspace{1pt}

What is of interest is the maximum amplitude of modulation at a specific location because the modulation will vary with the time between zero and maximum. To calculate this amplitude across all directions, the analysis in Grossman et al.\cite[page 20]{Grossman2017} can be better supported by the analysis conducted in Rampersad et al.\cite[section 2.5]{Rampersad2019}. Based on the two publications mentioned above, a complete description will be given here. The formula to calculate the maximum modulation amplitude along all directions at a specific location, $\vec{r} = (x,y,z)$, is:
\begin{equation}
    \label{eq:max_mod_amplitude}
    \vec{E}_{AM}^{max}(\vec{r}) = \begin{cases}
        2\big|\vec{E_2}\big| & \text{if}\; \big|\vec{E_2}\big| < \big|\vec{E_1}\big|\cos\alpha \\
        &\\
      2\dfrac{\Big|\big|\vec{E_2}\big|\times\big(\vec{E_1} - \vec{E_2}\big)\Big|}{\big|\vec{E_1} - \vec{E_2}\big|} & \text{otherwise}
    \end{cases}
\end{equation}
where $\alpha$ is the angle between $\vec{E_1}$ and $\vec{E_2}$, while \autoref{eq:max_mod_amplitude} holds only if $\alpha < 90\si{\degree}$. Whenever $\alpha \geq 90\si{\degree}$, the sign of one of the two fields can be flipped \textit{(it must be done consistently)}, since reaching peak field strength at different time points across different areas is what makes the $< 90\si{\degree}$ rule to be violated. This change is possible considering that \autoref{eq:max_mod_amplitude} calculates the maximum effect over one oscillation so that the overall effect is taken. The calculation of $\vec{E}_{AM}^{max}$ can be seen in \autoref{alg:max_modulation_amplitude} in \autoref{appndx:algorithms}.

% \input{Sections/electrical_stimulation_methods.tex}
% \pagebreak
\chapter{Transcranial Temporal Interference Stimulation (tTIS) Theory}

To better understand the physics behind the simulations, a brief explanation regarding the conduction within the media is presented in \autoref{sec:e_ohmic_qs}, followed by an analysis regarding the \gls{tTIS} hypothesis based on Grossman et al.\cite{Grossman2017} work.

\section{Electric Field Ohmic Quasi-static Approximation}
\label{sec:e_ohmic_qs}

Generally Maxwell's equations for electromagnetic wave propagation in a medium are as follows:
% Maxwells equations
\begin{center}
\begin{minipage}{.35\linewidth}
    \begin{equation}
        \nabla\cdot\vec{E}=\dfrac{\rho}{\epsilon}
    \end{equation}
\end{minipage}
\begin{minipage}{.35\linewidth}
    \begin{equation}
        \nabla\cdot\vec{B} = 0
    \end{equation}
\end{minipage}\break
\begin{minipage}{.35\linewidth}
    \begin{equation}
        \label{eq:maxwell_curl_e}
        \nabla\times\vec{E}=-\dfrac{\partial\vec{B}}{\partial t}
    \end{equation}
\end{minipage}
\begin{minipage}{.35\linewidth}
    \begin{equation}
        \nabla\times\vec{B} = \mu\Bigg(\vec{J} + \epsilon\dfrac{\partial\vec{E}}{\partial t}\Bigg)
    \end{equation}
\end{minipage}
\end{center}

\noindent The problem in question, finding the electrical field distribution in a volume using low frequencies, can be approached by simplifying the general form of Maxwell's equations and deriving the Quasi-static approximation format. The first step is to define the assumptions taken in order for such an approach to be valid.

For the frequencies used in the studied problem, in \si{kHz} range, the displacement current can be neglected making the Ohmic currents dominate. Also, since the magnetic field is not time variant, based on \autoref{eq:maxwell_curl_e} we can write:
\begin{equation}
    \label{eq:curl_zero_e_field}
    \nabla\times\vec{E} = \vec{0}
\end{equation}
and as we know when a field is irrotational then it can be calculated from a scalar potential ($\phi$) as seen below:
\begin{equation}
    \label{eq:e_field_from_potential}
    \boxed{\vec{E} = -\nabla\phi}
\end{equation}

\noindent Moreover, since the sum of currents entering and exiting the volume is zero \textit{(Kirchhoff's second law)} we can denote:
\begin{equation}
    \nabla\cdot\vec{J} = 0
\end{equation}
where here $\vec{J}$ is the ohmic current seen below as it is assumed that there are no current sources in the volume:
\begin{equation}
    \label{eq:sigma_e_0}
    \vec{J} = \sigma\vec{E}\Rightarrow\boxed{\nabla\cdot\big(\sigma\vec{E}\big) = 0}
\end{equation}
with $\sigma$ being the electrical conductivity of each medium. Finally, based on \cref{eq:e_field_from_potential,eq:sigma_e_0} the final relationship describing the problem can be derived:
\begin{equation}
    \label{eq:laplace_e}
    \boxed{\nabla\cdot(\sigma\nabla\phi) = 0}
\end{equation}

\autoref{eq:laplace_e} describes the problem of conduction, using bulk conductors, but it is worth noting that the equation is only valid in the frequency domain where the displacement current effects are negligible and only when there is no charge generated \textit{(charge is conserved)}. Furthermore, care shall be taken with the conductivity values, since depending on the material it may be dependant on the frequency used for the stimulation.

\pagebreak
\section{Temporal Interference}
The utilization of \gls{tTIS} to achieve targeted \gls{DBS} was first introduced by Grossman et al.\cite{Grossman2017}. This technique takes advantage of the spatial electromagnetic wave interference, using frequencies at the \si{kHz} range, having almost no effect on neurons since it is known that they do not respond to higher than 1\si{kHz} \draft{(citation needed)} frequencies.
\\\vspace{1pt}

\begin{wrapfigure}{r}{0.48\textwidth}
    \vspace{-10pt}
    \centering
    \includegraphics[width = 0.44\textwidth]{assets/images/brain_figure_ttis.pdf}
    \caption[Depiction of the \gls{tTIS} pattern and the vector direction of the electric field. The purple area is the \gls{ROI} where interference happens.]{Depiction of the \gls{tTIS} pattern and the vector direction of the electric field. The purple area is the \gls{ROI} where interference happens. Image by \href{https://pixabay.com/users/openclipart-vectors-30363/?utm_source=link-attribution&amp;utm_medium=referral&amp;utm_campaign=image&amp;utm_content=150935}{OpenClipart-Vectors} from \href{https://pixabay.com/?utm_source=link-attribution&amp;utm_medium=referral&amp;utm_campaign=image&amp;utm_content=150935}{Pixabay}}
    \label{fig:brain_elec_demo}
\end{wrapfigure}

Using two pairs of electrodes (\autoref{fig:brain_elec_demo}), with each pair having a slightly different frequency, an interference pattern can be generated in the conducting medium oscillating at the difference of the two frequencies. Depending on the nature of the medium the pattern can vary and as illustrated in Grossman et al.\cite{Grossman2017} when a uniform medium is used, it is simple and easy to calculate.

Since the modulation happens in the 3D space, there will be different patterns in the $x$, $y$ and $z$ directions. According to Grossman et al.\cite[page 20]{Grossman2017}, at any location $\vec{r} = (x,y,z)$ the envelope amplitude of the \gls{AM} of the electric field produced by the temporal interference, it is calculated as:
\begin{equation}
    \label{eq:directional_amplitude}
    \vec{E}(\vec{n},\vec{r}) = \Big|\big|(\vec{E_1} + \vec{E_2})\cdot\vec{n}\big| - \big|(\vec{E_1} - \vec{E_2})\cdot\vec{n}\big|\Big|
\end{equation}
where $\vec{E_1} = \vec{E_1}(\vec{r})$, $\vec{E_2} = \vec{E_2}(\vec{r})$ are the electric fields coming from the two electrodes and $\vec{n} = \vec{n}(\vec{r})$ is the unit vector at the direction of interest.
\\\vspace{1pt}

What is of interest is the maximum amplitude of modulation at a specific location since the modulation will vary with time between zero and maximum. To calculate that amplitude across all directions, the analysis on Grossman et al.\cite[page 20]{Grossman2017} can be better supported by the analysis done on Rampersad et al.\cite[section 2.5]{Rampersad2019} and based on the two aforementioned publications, a complete description will be given here. The formula to calculate the maximum modulation amplitude, along all directions at a specific location, $\vec{r} = (x,y,z)$, is:
\begin{equation}
    \label{eq:max_mod_amplitude}
    \vec{E}_{AM}^{max}(\vec{r}) = \begin{cases}
        2\big|\vec{E_2}\big| & \text{if}\; \big|\vec{E_2}\big| < \big|\vec{E_1}\big|\cos\alpha \\
        &\\
      2\dfrac{\Big|\big|\vec{E_2}\big|\times\big(\vec{E_1} - \vec{E_2}\big)\Big|}{\big|\vec{E_1} - \vec{E_2}\big|} & \text{otherwise}
    \end{cases}
\end{equation}
where $\alpha$ is the angle between $\vec{E_1}$ and $\vec{E_2}$, while \autoref{eq:max_mod_amplitude} holds true only if $\alpha < 90\si{\degree}$. Whenever $\alpha \geq 90\si{\degree}$, the sign of one of the two fields can be flipped \textit{(it must be done is a consistent fashion)} since reaching peak field strength at different time points across different areas, is what makes the $< 90\si{\degree}$ rule to be violated. Such a change is possible considering that \autoref{eq:max_mod_amplitude} calculates the maximum effect over one oscillation, so the overall effect is the one that we care. The calculation of $\vec{E}_{AM}^{max}$ can be seen in \autoref{alg:max_modulation_amplitude} at \autoref{appndx:algorithms}.

\pagebreak
\chapter{Finite Element Modelling}

All models are discretized using a \gls{FE} mesh, before being passed to the solver. While there are numerous \gls{FEM} solver tools out there, the one being used in this work is \textit{SfePy} \cite{Cimrman2019}, since it provides all the necessary means for the solution of the problem in question and it interfaces directly with Python. In this section a detailed overview of the models used (\autoref{sec:fem_models}), the meshing strategy (\autoref{sec:fem_meshing}) and the solver configuration (\autoref{sec:fem_solver}) is provided.

\section{Models}
\label{sec:fem_models}

A tricky part in simulations involving the brain is to choose the correct or the most representative model for the desired work. While spherical models are used just for a proof of concept, very complex models on the other hand increase computational time a lot. This section describes the generation of the spherical model as well as the use of publicly available realistic human head models.

\subsection{Simple Models}
\begin{wrapfigure}{r}{0.35\textwidth}
    \centering
    \includegraphics[width = 0.34\textwidth]{assets/images/sphere_brain.pdf}
    \caption{Layers of the spherical model}
    \vspace{-4cm}
    \label{fig:sphere_brain}
\end{wrapfigure}

A good approximation of the layered structure of the human head, using the spherical model, is to use four layers in total with varying widths. These four layers as seen on \autoref{fig:sphere_brain} are the same ones described in \paper{Grossman}{Grossman2017}. To generate the spherical model Gmsh \cite{gmsh}, a 3D modeling software, was used with the layer radius values as shown in \autoref{tab:spherical_layers}. The script to generate these models can be seen in \autoref{lst:sphere_code} and it can also be found on the GitLab repository \cite{thesis_repo} as a snippet.
\begin{table}[!ht]
\begin{minipage}{.62\linewidth}
\centering
\caption{Layer radius of the sphere model. Base radius is considered $87\;\si{mm}$.}
\label{tab:spherical_layers}
\begin{tabular}{|c|c|c|}
\hline
\rowcolor[HTML]{C0C0C0} 
{\color[HTML]{000000} \textbf{Layer}} & {\color[HTML]{000000} \textbf{Radius {[}mm{]}}} \\ \hline
Skin & 85.96 \\ \hline
Skull & 81.61 \\ \hline
CSF & 74.21 \\ \hline
Brain & 72.21 \\ \hline
\end{tabular}
\end{minipage}
\end{table}

\noindent The radius of each layer \textit{(lower bound)} comes from the normalized layer thickness provided by Grossman et al.\cite[Figure S2, J and K]{Grossman2017}, having as an outer radius (base radius) the surface of the skin.

\begin{lstlisting}[language=C,caption={Sphere \gls{CAD} model generation code in \texttt{geo} format},captionpos=b, label=lst:sphere_code]
SetFactory("OpenCASCADE");

head_radius = 87;

// Ratio values from Grossman et al. 2017
brain_ratio = 0.83;
csf_ratio = 0.023;
skull_ratio = 0.085;
scalp_ratio = 0.05;

Sphere(1) = {0, 0, 0, head_radius};
Sphere(2) = {0, 0, 0, head_radius*(brain_ratio + csf_ratio + skull_ratio + scalp_ratio)};
Sphere(3) = {0, 0, 0, head_radius*(brain_ratio + csf_ratio + skull_ratio)};
Sphere(4) = {0, 0, 0, head_radius*(brain_ratio + csf_ratio)};
Sphere(5) = {0, 0, 0, head_radius*(brain_ratio)};

Physical Volume("Outer Boundary", 1) = {1};
Physical Volume("Skin", 2) = {2};
Physical Volume("Skull", 3) = {3};
Physical Volume("CSF", 4) = {4};
Physical Volume("Brain", 5) = {5};

Physical Surface("Outer Boundary", 1) = {1};
Physical Surface("Skin", 2) = {2};
Physical Surface("Skull", 3) = {3};
Physical Surface("CSF", 4) = {4};
Physical Surface("Brain", 5) = {5};

Mesh 2;
RefineMesh;
RefineMesh;
RefineMesh;

Mesh 3;
RefineMesh;
\end{lstlisting}

\subsection{Realistic Human Head Models}
\label{sec:phm_models}

Realistic brain models are really hard to draw in a \gls{CAD} program and be anatomically accurate, so they are usually generated using medical imaging data from \gls{MRI} and \gls{CT} scans. The models used in the simulations of this work are taken from the \gls{PHM} repository of \gls{IT'IS} \cite{ErikG.Lee2016,Lee2018,ITstissue}. Each model contains a total of 7 (seven) 3D triangular surface meshes \textit{(skin, skull, \gls{CSF}, grey matter, white matter, cerebellum and ventricles)} in \gls{STL} format, which were generated from \gls{MRI} data. Sometimes the surface meshes are called \gls{BE} meshes and from now on this term might be used interchangeably with the \gls{STL} format.

There are also other ways to model the brain and the head, by directly utilizing \gls{MRI} data and generating the corresponding tetrahedral mesh from the medical imaging segmentation. One streamlined and easy way to achieve this is by using \texttt{Brain2Mesh}, a tool developed by Tran et al.\cite{Tran2020}, which can generate tetrahedral meshes from segmented \gls{MR} images.

\section{Meshing}
\label{sec:fem_meshing}

Correct meshing is key to obtaining correct results from the solution of a \gls{FEM} problem, so in this section the mesh quality criteria are described, as well as a basic mesh convergence analysis is presented. Apart from the analysis the meshing strategy followed is presented in detail, with specific code examples and the algorithms developed and used. For all meshing operations the tool in use is PyMesh \cite{pymesh}, which is includes all the necessary \gls{STL} file and geometry operations used in this work. Specifically for all volume meshing TetGen \cite{tetgen}, a software library specifically targeted in tetrahedral meshing, is used throughout this work.

Before proceeding with the detailed explanation of each process involved in the final meshed model generation, a high level diagram of the steps involved can be seen in \autoref{fig:mesh_modelling}. The order of the explanation will be in-line with this diagram.

\begin{figure}[H]
    \centering
    \includegraphics[width = \textwidth]{assets/images/fem_meshing_high_level.pdf}
    \caption{High level description of the \gls{FEM} meshing procedure}
    \label{fig:mesh_modelling}
\end{figure}

\noindent Each step in \autoref{fig:mesh_modelling} describes a single process, except from the annotation steps. To recognize each mesh domain from the meshed model and then annotate it, is a key process allowing different material definition later in the solver. During the meshing procedure a volumetric mesh is generated with no labels on each individual domain, thus the need to recognize each region \textit{(domain)} arises. The recognition and subsequent labeling is accomplished by splitting the domains with the help of the boundary surfaces and finding the enclosed tetrahedra, as described in \ref{subsec:elec_annotation}.

Finally, as it is evident from \autoref{fig:mesh_modelling} the meshed model, with all required annotations included, is saved in \gls{VTK} format which is a very common and robust file format for \gls{FE} meshes.

\subsection{Data Required}

The first step in the meshing pipeline is to import all the necessary files for the geometry and the electrode placement. The geometry files from the \gls{PHM} repository \cite{ErikG.Lee2016} are loaded as \gls{STL} format, which contains the surface boundaries of each region, as described in \ref{sec:phm_models}.

Apart from the geometry files, the electrode coordinate files are also loaded. The electrode coordinates are generated using \texttt{Mesh2EEG} \cite{Giacometti2014}, a tool developed by Thayer School of Engineering at Dartmouth \cite{mesh2eeg_web}, and more details are provided in \ref{subsec:elec_placement}.

\subsection{Electrode Array Generation \& Placement}
\label{subsec:elec_placement}

The term electrode array generation means the creation of the electrode cluster from the coordinates to having the actual \gls{FE} meshes of each individual electrode. Securing repeatability across different heads \textit{(varying geometry)} is key to having comparable results and drawing conclusions, thus the standard electrode positioning systems widely used in \gls{EEG} studies have been utilized here too. In particular the \textbf{10-20 international system} \cite[chapter 13]{Malmivuo1995} is used throughout this work, as it is the most common system in the clinical setting. 

To understand where the naming convention \textbf{10-20} comes from, it is mandatory to define the landmarks used. These are the \textit{\gls{nasion}}, \textit{\gls{inion}}, \textit{left} and \textit{right \gls{preauricular}s} and define the start of measurement for the electrode placement on the head. The numbers in the name of the system refer to the percentile arch length of the arch formed by the \gls{nasion} and \gls{inion}, as well as the left and right \gls{preauricular}s. More specifically the number \textbf{10} denotes that the placement shall start at $10\%$ of the arch length, while the number \textbf{20} denotes the arch length that each individual electrode shall have, after the placement of the first one. Denser systems than the \textbf{10-20} exist, namely the \textbf{10-10} and \textbf{10-5}, following the same placement rules.Lastly, it should be noted that the electrodes are placed in a circular fashion and to aid in the visualization of the placement, some details are shown on \autoref{fig:10_20_explanation}.

\begin{figure}[H]
    \centering
    \includegraphics[width = 0.85\textwidth]{assets/images/10-20_explanation.png}
    \caption[Landmarks of the 10-20 international \gls{EEG} electrode positioning system]{Landmarks of the 10-20 international \gls{EEG} electrode positioning system. \cite[figure 13.2, p.368]{Malmivuo1995}}
    \label{fig:10_20_explanation}
\end{figure}

As in real life, to calculate the individual electrode position the landmarks have to identified. This process is done manually by reviewing the 3D geometry of the each model and identifying the 4 (four) required landmarks. Having the coordinates of each landmark, in the coordinate system of the \gls{STL} file in question, the positions can be easily calculated with the help of the \texttt{Mesh2EEG} \cite{Giacometti2014} tool, running on \gls{MATLAB}. This tool takes as an input the 3D surface mesh in \gls{STL} format and the coordinates of the four landmarks on the model. The output then is the coordinates of each point in all three aforementioned systems, the \textbf{10-20}, \textbf{10-10} and the \textbf{10-5}.

Using the output coordinates from the \texttt{Mesh2EEG} \cite{Giacometti2014} tool, the final step for creating the actual electrodes is to generate a cylindrical mesh for each individual one by using the calculated coordinates as the center point and orienting the cylinder normally to the surface of the head \textit{(skin)}.
\\\vspace{1pt}

\begin{wrapfigure}{l}{0.48\textwidth}
    \centering
    \vspace{-15pt}
    \includegraphics[width = 0.45\textwidth]{assets/images/eeg_electrodes_10-20.pdf}
    \caption[10-10 system names. The orange electrodes are used in the 10-20 system.]{10-10 system names. 10-20 system depicted in orange. \href{http://www.mariusthart.net/downloads/eeg_electrodes_10-20.svg}{Illustration} by \href{http://www.beteredingen.nl}{Marius 't Hart} licensed under \href{http://creativecommons.org/licenses/by-sa/3.0/nl/deed.en_GB}{CC BY-SA v3.0}}
    % \vspace{-0.5cm}
    \label{fig:electrodes_10-20}
\end{wrapfigure}

Upon the generation of the electrode array, placing them on the head surface is next, one of the most important factors of making \gls{tTIS} work properly. The positioning affects the stimulation pattern inside the brain, thus selecting the right electrodes, one can achieve the desired stimulation.

The spatial distribution of the electrodes is shown in \autoref{fig:electrodes_10-20} and the orange highlighted positions showcase the \textbf{10-20} system electrodes. Before the whole mesh can be generated, each electrode has to be oriented such that it is normal at the area of interest, where the orientation is calculated based on \autoref{alg:electrode_orientation} at \autoref{appndx:algorithms}. Finally, the electrode placement on the surface of the head model is completed with the successful orientation of all electrodes and the consequent surface mesh merge of the electrode array and the skin.

For the electrodes to be usable by the solver later they have to be annotated, meaning a number \textit{(ID)} shall be assigned to each individual electrode. The details of the annotation procedure are discussed in \ref{subsec:elec_annotation}.

\subsection{Mesh Generation}
\label{subsec:mesh_generation}

For each model the tetrahedral mesh was generated using TetGen \cite{tetgen}, which provides the ability to specifically choose the parameters necessary for the modeling. TetGen \cite{tetgen} is a C++ program that can generate tetrahedral meshes from any 3D surface provided. To control the mesh quality, TetGen \cite{tetgen} uses the following parameters:
\begin{itemize}
	\item the \textit{radius-to-edge-ration} $(q)$
	\item the maximum tetrahedron volume $(V_{max})$
\end{itemize}
with $q$ being the ratio of the circumference and the shortest edge length of each tetrahedron, while $V_{max}$ is the maximum allowable volume that an element can have. All the quantities are in the units each model is designed with. Specifically, the command line arguments used are shown in \autoref{lst:tetgen_commands}. More details on the software compilation and the integration with all the other tools, can be found in \autoref{chap:soft_arch}.

\begin{lstlisting}[language=bash,caption={Command to run TetGen},captionpos=b, label=lst:tetgen_commands]
	./tetgen -zpq1.414/0O4a30kANEFVV filename_merged.stl
\end{lstlisting}

As it can be seen in \autoref{fig:mesh_comparison}, the mesh generated with $q = 1.414$ has almost double the amount of elements than the one generated with $q = 2.0$. The difference can be clearly seen in the green highlighted circular areas of \autoref{fig:mesh_comparison}, where it is prevalent that the mesh captures better the boundary surface and the number of elements at this area is greater with $q = 1.414$ compared to the $q = 2.0$ case. The $q$ and $V_{max}$ presented here for the dense ($q=1.414$) mesh, are inspired by Tran et al.\cite{Tran2020}.

\begin{figure}[H]
    \centering
    \includegraphics[width = 0.85\textwidth]{assets/images/mesh_comparison.pdf}
    \caption[Mesh quality comparison for the different $q$ parameters.]{Mesh quality comparison of the different mesh parameters. The green circles highlights an area where differences are clearly visible, in regards to mesh quality and number of elements.}
    % \vspace{-1.2cm}
    \label{fig:mesh_comparison}
\end{figure}

Having better mesh quality always comes at a cost of computational time both for the mesh generation and the subsequent solution, as well as computational resources. Meshing with too many elements can prohibit the solver from running even on average desktop computers due to very high \gls{RAM} requirements. Keeping the computational time and resources in mind, there is always a trade-off to select the best balance between the desired and the actual solution. Such an assessment is conducted through a mesh convergence study in which the problem is solved for the same model by varying the mesh parameters in each run. Details about the mesh convergence run for this work can be found in \ref{sec:mesh_quality}.

Before assessing the mesh quality by solving the problem, there are some ground rules to be followed for a good mesh. One of those rules is that the \gls{tetrahedron aspect ratio}, the ratio between the circumference of each tetrahedron and the shortest edge length, should be as low as possible, meaning that more symmetric tetrahedra will exist. Lowering this ratio has a direct impact on another metric, the dihedral angle. Comparing the graphs in \autoref{fig:mesh_quality_graphs} for the two different $q$ values, it is evident that the mesh with the lower $q$ is better, since the \gls{tetrahedron aspect ratio} is accumulated in lower valued areas. This ratio is key as the lower the value the more symmetric each tetrahedron is, thus eliminating potential degenerate elements in the mesh.

\begin{figure}[H]
    \centering
    \includegraphics[width = \textwidth]{assets/images/mesh_quality_graphs.pdf}
    \caption[Mesh quality graphs for each different $q$ parameter]{Mesh quality graphs for each different $q$ parameter. The mesh statistics were generated by TetGen \cite{tetgen}.}
    % \vspace{-1.2cm}
    \label{fig:mesh_quality_graphs}
\end{figure}

To complete the mesh quality assessment, the mesh convergence study is presented in \autoref{sec:mesh_quality} and the meshing values for this work are chosen.

\subsection{Region \& Electrode Annotation}
\label{subsec:elec_annotation}

The final step before having the mesh elements ready to be imported by the solver is the annotation or labelling of the mesh entities. This process is easy when using the \texttt{-A} TetGen \cite{tetgen} switch, which automatically assigns region attributes based on the lower boundary of each one. For the realistic head models the given region attributes can be seen in \autoref{tab:domain_annotation}.

\begin{table}[!ht]
	\centering
	\caption{Mesh region labels for each mesh entity in the realistic human head models}
	\label{tab:domain_annotation}
	\begin{tabular}{|c|c|}
		\hline
		\rowcolor[HTML]{C0C0C0} 
		{\color[HTML]{000000} \textbf{Layer}} & {\color[HTML]{000000} \textbf{Number}} \\ \hline
		White Matter & 1 \\ \hline
		Grey Matter & 2 \\ \hline
		\gls{CSF} & 3 \\ \hline
		Skull & 4 \\ \hline
		Skin & 5 \\ \hline
		Cerebellum & 6 \\ \hline
		Ventricles & 7 \\ \hline
	\end{tabular}
\end{table}

Regarding the electrode labels the count starts from \textbf{10} \textit{(assigned to Fp1)} and the direction of labelling is from \gls{nasion} to \gls{inion} and from left to right, with the directions as shown in \autoref{fig:electrodes_10-20}.

\subsection{Viewing the Mesh}

The visual assessment of the mesh quality and the visualization of the solution, can be done using Paraview \cite{paraview} which supports the \gls{VTK} file format used in this work for all meshed models.

\section{Solver}
\label{sec:fem_solver}

After successfully meshing the model, the next step is to solve the problem numerically using a \gls{FEM} solver. The chosen package, containing a plethora of solvers, is SfePy \cite{Cimrman2019}. In the following sections an overview about the boundary conditions used, as well as the different solvers tested will be provided.

Getting to a solution is a multi-step process and the specifics steps are shown in \autoref{fig:solver_high_level}. The mesh loaded here is the one generated with the procedure described in \ref{subsec:mesh_generation}.

\begin{figure}[H]
    \centering
    \includegraphics[width = \textwidth]{assets/images/fem_solver_high_level.pdf}
    \caption{High level description of the \gls{FEM} solver procedure}
    \label{fig:solver_high_level}
\end{figure}

\subsection{Loading the Mesh \& Settings}

There are two important files to be loaded in the solver. One is the meshed model in \gls{VTK} format and the other is the settings for the simulation in \gls{YAML} format. The settings file includes the path of the meshed model, the path for the different scripts and the region labels for both the model regions (e.g. brain, skin), as well as the electrode regions, with the corresponding conductivity values per region.

It should be noted that if the built-in mesh importer from SfePy is used, the region labels array in the meshed model file shall be named as \texttt{mat\_id}, otherwise the labels are not imported. For the purposes of this work, this has been abstracted and more details regarding the implementation can be found in \autoref{chap:soft_arch}.

\subsection{Meshing Impact}
\label{sec:mesh_quality}

As discussed in \ref{subsec:mesh_generation}, having better mesh quality helps to outline detailed shapes. Not only the outline is important, but also the actual field distribution upon solution. It is clearly visible from \autoref{fig:solved_mesh_comparison} that having a more fine mesh produces a better result in the problem's solution. The main question is if this difference is significant or not.

\begin{figure}[H]
    \centering
    \includegraphics[width = 0.85\textwidth]{assets/images/solved_mesh_comp.pdf}
    \caption[Problem solution comparison for the different $q$ mesh parameters.]{Problem solution comparison of the different mesh parameters. The green circles highlights an area where differences are clearly visible, in regards to mesh outline and number of elements.}
    \label{fig:solved_mesh_comparison}
\end{figure}

In the case of the presented $q$ values on \autoref{fig:solved_mesh_comparison}, the difference between the field distribution is prominent. The mesh with a lower $q$ value, meaning more tetrahedra are closer to being regular, has a smoother field distribution hence this difference can be considered significant compared to the $q=2.0$ mesh. One way to find a suitable value is through reviewing the changes between different runs of a solved model with varying $q$ values. In \draft{fig:mesh\_conv\_study} a solved model with different mesh densities can be found.

\draft{Add the figure of the mesh convergence study.}

Progressing with $0.1$ step size from $q=2.0$ to $q=1.0$ on \draft{fig:mesh\_conv\_study}, it can be seen that up to \draft{q=1.6} there is a difference between the patterns of the solved model, while below \draft{q=1.4} there is no visible difference in the field distribution. To find the suitable value, the next step is to repeat the process of solving with varying $q$ values, but now with a finer step in the region that has been identified as "converged" \textit{(no significant change from one value to the next)}. The final range shall be tested on at least one more model and if there is less than \textbf{2\%} difference in the values obtained from the solver, then the mesh can be considered converged and the final $q$ value is chosen. However for simplicity reasons here the value of $\boldsymbol{q=1.5}$ will be used for all the model meshes.

\subsection{Solver Set-up \& Selection}

One of the important factors in having a successful solution of the model is using the correct solver for the problem. Depending on the mathematical nature of the physical problem, different solvers can give more accurate results and in shorter time.

Generally there are two major solver categories, the direct and the iterative each of which is explained below. In this work an iterative solver is the most suitable with the specific algorithms presented in \ref{subsec:iterative_solvers}, since as described in \ref{sec:e_ohmic_qs} the problem faced here is solving the Laplace equation, a second-order partial differential equation.

\subsubsection{Direct Solution}

The most accurate solution of a \gls{FEM} problem is evaluated through a direct solver. This kind of solvers try to approach the solution by using traditional linear algebra techniques, like inverse matrix calculation, to find the matrix of unknowns. Direct solvers can take up huge amounts of \gls{RAM} which makes them unsuitable for problem with a few million elements since complete access to the whole matrix is required. In the case of realistic brain models the element count can sometimes reach the order of 6 million in a square matrix. Apart from the \gls{RAM} issues, these solvers are not suitable since most of the times the \gls{FE} matrices are heavily sparse, meaning most of their elements are zero, thus having many unnecessary operations resulting to zero.

As it is obvious direct solvers are out of question for realistic brain models, since they can only be used on a grid of computers. Just to benchmark the performance, a direct solver called \gls{UMFPACK} \cite{Davis2004_umfpack} was used to solve the \texttt{126325} model from the \gls{PHM} repository \cite{ErikG.Lee2016}. Results are summarized below on \autoref{tab:direct_solver_bench}. All the direct solver benchmarks were obtained though running the code on the \gls{HPC} grid of \gls{AUTh} \gls{IT} center.

\begin{table}[!ht]
	\centering
	\caption{\gls{UMFPACK} direct solver benchmark using the \texttt{126325} \gls{PHM} model}
	\label{tab:direct_solver_bench}
	\begin{tabular}{|c|c|}
		\hline
		\rowcolor[HTML]{C0C0C0} 
		\textbf{Parameter} & \textbf{Value} \\ \hline
		Matrix Size & $2260222\times 2260222$ \\ \hline
		Sparsity (Fill) & $6.63\cdot 10^{-6}\%$ \\ \hline
		Execution Time & 30 min \\ \hline
		RAM Usage & 125 GB \\ \hline
	\end{tabular}
\end{table}

\subsubsection{Iterative Solution}
\label{subsec:iterative_solvers}

The logic behind this type of solvers is that initially a guess is made for the solution, following iterations afterwards to find the exact solution based on the provided convergence criteria.

In this work iterative solvers were used and specifically the \gls{CG} solver. Before solving the actual problem the matrix first is passed through a preconditioner which transforms the matrix in a more suitable form for numerical solution. Here the \gls{HYPRE} \cite{hypre-web-page} suite was used with the Boomer\gls{AMG} \cite[chapter 4]{McCormick1987_amg} preconditioner. The multigrid method is used in solving partial differential equations by hierarchical discretization of the problem. Algebraic Multigird specifically is construct the hierarchy from the system matrix instead of directly using the differential equations. 

Furthermore, to make the system matrix less heavy in terms of elements a coarsening algorithm is utilized along with the preconditioner and that is the \gls{HMIS}, which again belongs in the \gls{HYPRE} family of tools. All these packages are included in the python version of the \gls{PETSc} \cite{petsc-web-page,petsc-user-ref,petsc-efficient}, called \texttt{petsc4py} \cite{Dalcin2011}. \gls{PETSc} is a toolset which can be used to parallelize the problems and includes many solvers for sparse matrices \textit{(matrices with most elements zero)}, as this is the case for our problem.

\subsection{Boundary Conditions}
\label{subsec:solver_boundary_conditions}

Through all simulations, the Neumann boundary condition \textit{(flux through a surface)} $\big(\sigma\nabla\phi\cdot\vec{n} = 0\big)$ is used on the whole domain of and the Dirichlet boundary condition \textit{(potential on a surface)} $\big(\phi = \phi_0$, $\phi = -\phi_0\big)$ is used in the active electrode and the non-active electrode accordingly. The magnitude of $\phi_0$ is determined by the desired injected current.

\subsection{Post-processing Steps}

To get the desired results at the end of the problem solution, there is the need to further process the solved equations.

Solving the Laplace equation yields the potential at each node and to calculate the electric field the derivative of the potential is taken as:
\begin{equation}
	\vec{E} = -\nabla\phi
\end{equation}
where $\phi$ is the potential as solved by the \gls{FEM} solver. The combination if two different solutions, with different electrode pairs, is what is of interest and after calculating the potentials for each case, the electric field of the maximum as well as the directional modulation envelope is calculated using the \autoref{alg:max_modulation_amplitude} found in \autoref{appndx:algorithms}.

\pagebreak
\chapter{Software Architecture}
\label{chap:soft_arch}

Finding the right software and the right tools for the specific problem requirements is usually a challenging task. There are many proprietary software available providing an intuitive user interface, and the required tools for \gls{FE} analysis. Apart from the proprietary ones there are many other open-source software like \gls{SimNIBS}, and SCIRun. The challenge is to combine the weaknesses of one software with the strengths of the other.

In this work an interface-like software has been developed for the simulations of the \gls{tTIS} method. The boundaries are not strictly set for this purpose only, but the software in question is easily scalable to multiple physics problems requiring \gls{FEM} solution. As mentioned before, this work uses SfePy \cite{Cimrman2019}, PyMesh \cite{pymesh}, and TetGen \cite{tetgen} packages. These are all open-source and actively developed, with TetGen \cite{tetgen} being highly used in application involving tetrahedral meshing.

As part of experimentation and learning, custom functions annotating and recognizing the regions of each model were developed and can be found on GitLab \cite{thesis_repo}, at \href{https://gitlab.com/ttis-simulations/ttis-software/-/releases#v0.1}{v0.1 release}. 

\section{UML Class Diagram}

Better understanding of the software's structure and function requires using \gls{UML} class and use case diagrams, highly established ways of technical communication in software engineering.

\autoref{fig:uml_class_meshing} presents the class relations and definitions of the meshing procedure. The steps taken to mesh the model are illustrated in \autoref{fig:use_case_meshing}.

\begin{figure}[H]
    \centering
    \includegraphics[width = 0.85\textwidth]{assets/images/uml_class_diagram_meshing.png}
    \caption{\gls{UML} class diagram for this work's meshing classes}
    \label{fig:uml_class_meshing}
\end{figure}

The meshing procedure is a bit involved, in terms of steps and packages required, since it generates a TetGen file which is the meshed using TetGen. The \gls{FEM} solution procedure is much simpler, with most of the functionality done by the \texttt{Solver} class automatically. The only required package for the normal operation of the solver is SfePy. The \texttt{Solver} class is highly modular and can be modified to suit the needs of the problem, by just altering the equation and the boundary condition definition.

\begin{figure}[H]
    \centering
    \includegraphics[width = 0.38\textwidth]{assets/images/uml_class_diagram_fem.png}
    \caption{\gls{UML} class diagram for this work's \gls{FEM} class}
    \label{fig:uml_class_fem}
\end{figure}

\section{Use-case diagrams}

An overview of the meshing procedure is provided in \autoref{fig:use_case_meshing}. This relies on the class implementation (\autoref{fig:uml_class_meshing}) and the relevant code executing the use cases described, can be found in \autoref{appndx:code}.

\begin{figure}[H]
    \centering
    \includegraphics[width = 0.76\textwidth]{assets/images/uml_meshing_use_case_diagram.png}
    \caption{\gls{UML} use case diagram for the meshing procedure}
    \label{fig:use_case_meshing}
\end{figure}

Same as above, \autoref{fig:use_case_fem} illustrates the procedure to solve the meshed model. The code regarding these use cases can be found in \autoref{appndx:code}.

\begin{figure}[H]
    \centering
    \includegraphics[width = 0.76\textwidth]{assets/images/uml_fem_use_case_diagram.png}
    \caption{\gls{UML} use case diagram for the \gls{FEM} solution procedure}
    \label{fig:use_case_fem}
\end{figure}

\pagebreak
\chapter{Simulation Results}

The entirety of this work is based on the method introduced by \paper{Grossman}{Grossman2017}, which has been tested on murine brains and the results where very close with the simulated ones. Although on murine brain the method can produce the desired results, on the human brain it is much more difficult to achieve the required outcome and much more difficult to accurately simulate it.

\paper{Grossman}{Grossman2017} simulation models were based on a simple spherical layered model, with uniform density of the material. Such an approach is working for small brains, like murine brain, but it fails for human models where the structure is very complex and greatly varying between subjects. One way to better approach the problem on humans is to use detailed models, capturing the non-uniformities and geometric details, giving results closer to reality.

The scope of this work is to try and investigate the potential use of the \paper{Grossman}{Grossman2017} method on humans, while noting the road blocks of accurate simulation. Furthermore, since the method's effectiveness is heavily based on the injected current ratio and the electrode pair positioning, a discussion is provided in \autoref{sec:discussion} for potential optimization utilization. In the current section \paper{Grossman}{Grossman2017} approach with the spherical model is used as a benchmark for the validation of the software developed and the method is further extended for human models, where the results are seen in \autoref{sec:realistic_human_models}.

\section{Simple Spherical Layered Model}

The simple model is comprized of 4 layers in total, with varying thickness each. The layer thickness is included in \autoref{tab:spherical_layers} as a radius fraction and the values for each layer come from Grossman et al. \cite[Figure S2, J and K]{Grossman2017} \textit{(referred hereafter as paper)}. Additionally, the conductivity values used in the paper are summarized in \autoref{tab:grossman_conductivity_vals}.

\begin{table}[!ht]
\centering
\caption{Conductivity values for the simple model taken from \cite{ITstissue}}
\label{tab:grossman_conductivity_vals}
\begin{tabular}{|c|c|c|}
    \hline
    \rowcolor[HTML]{C0C0C0} 
    {\color[HTML]{000000} \textbf{Layer}} & {\color[HTML]{000000} \textbf{Conductivity {[}S/m{]}}} \\ \hline
    Skin & 0.17 \\ \hline
    Skull & 0.003504 \\ \hline
    CSF & 1.776 \\ \hline
    Brain & 0.234 \\ \hline
\end{tabular}
\end{table}

To benchmark this work's software the same configurations as in the paper were used and the results can be seen in \autoref{fig:grossman_thesis_comparison}.

\begin{figure}[H]
    \centering
    \begin{subfigure}[b]{0.49\textwidth}
        \centering
        \includegraphics[width = \textwidth]{assets/images/grossman_sphere_electric_field.png}
        \caption{Figure 2B(i) from \paper{Grossman}{Grossman2017}}
        \label{fig:grossman_envelope}
    \end{subfigure}
    \begin{subfigure}[b]{0.49\textwidth}
        \centering
        \includegraphics[width = \textwidth]{assets/images/grosman_benchmark.png}
        \caption{Solution of this work's software}
        \label{fig:envelope_at_y_benchmark}
    \end{subfigure}
    \caption[Modulation envelope on the y-direction for the spherical layered model]{Modulation envelope on the y-direction for the spherical layered model, same as on Gross et al. \cite[Figure 2B]{Grossman2017}.}
    \label{fig:grossman_thesis_comparison}
\end{figure}

It is clear from the figures above that the software can be considered accurate, based on the paper's results, so the analysis can proceed on the realistic human head models in the following section.


\section{Realistic Human Models}
\label{sec:realistic_human_models}

Going from a spherical model, to a murine model and then to a realistic human model is great leap from simplicity to great complexity. To be able and test any hypothesis in clinical trials and get validated, first the processes of modelling, testing on murine models and phantoms have to precede.

As it has been explained so far, \gls{tTIS} shows great potential to be used for non-invasive \gls{DBS}, thus the leap from the simple spherical model of \paper{Grossman}{Grossman2017} to realistic human models must be made. Two key factors have been tested in this work and more specifically the results of the effect of the human head anatomy variation as well as the variation of the electrode position are presented in \cref{subsec:effects_of_model_anatomy,subsec:varying_electrode_position}. As it will be evident below, the anatomy of each model, i.e., of each person has a significant impact on the final pattern inside the brain, thus there is the need for optimization to achieve stimulation of a small area, as different electrode combinations must be used for the each person or model. The optimization is not included in this work as it is part of future work and currently it is at a preliminary stage, however a discussion is provide on \autoref{sec:discussion}.

Lastly, it should be noted that for the electrode positioning impact assessment, models with the international 10-10 system have also been tested to find the potential issues that the 10-20 coarse nature might have. It should be noted that all conductivity values for the simulated models are taken from the \gls{IT'IS} tissue database \cite{ITstissue}.

\subsection{Effects of Model Anatomy}
\label{subsec:effects_of_model_anatomy}

Assessing the impact of the model anatomy on the field distribution requires many different subjects to have statistically significant results. Specifically in this work, 9 models from the \gls{PHM} repository \cite{ErikG.Lee2016} were used, having the 10-20 as the electrode placement system. Electrodes P8, F7 \textit{(1\si{kHz})} and P7, F8 \textit{(1.04\si{kHz})} were used, with the first electrode in each couple being the active electrode. The corresponding electrode positions are shown in \autoref{fig:electrodes_10-20}.

\begin{figure}[H]
    \centering
    \begin{subfigure}[b]{0.49\textwidth}
        \includegraphics[width = \textwidth]{assets/images/y_axis_line_brain.png}
        \caption{Slice across the x-axis.}
        \label{fig:brain_slice_for_effects_x}
    \end{subfigure}
    \begin{subfigure}[b]{0.49\textwidth}
        \includegraphics[width = \textwidth]{assets/images/center_of_bounds_line_brain.png}
        \caption{Slice across the center of bounds.}
        \label{fig:brain_slice_for_effects_cf}
    \end{subfigure}
    \caption{Slice references for \cref{fig:x_axis_effect,fig:center_of_bounds_effect}. The white line represents the line of measurement.}
    \label{fig:brain_slice_for_effects}
\end{figure}

As a first step the maximum modulation envelope electric field value was taken across the line of the x-axis for each model, with the reference plane illustrated in \autoref{fig:brain_slice_for_effects_x}. The field values for each model, along the x-axis, can be seen in \autoref{fig:x_axis_effect}, along with the standard deviation in the legend.

\begin{figure}[H]
    \centering
    \includegraphics[width = \textwidth]{assets/images/x_axis_line.pdf}
    \caption{Field distribution for each \gls{PHM} model along the x-axis. The number near each model is the standard deviation for the distribution. The curves were smoothed using a moving average with a window size of 50 samples.}
    \label{fig:x_axis_effect}
\end{figure}

It is evident from the shape of the curves in \autoref{fig:x_axis_effect} that the same pattern is followed, but there is a significant variation from one model to the next, both in the curve's shape and the standard deviation. Furthermore, another measurement was made along the center of bounds axis (\autoref{fig:brain_slice_for_effects_cf}) and as illustrated in \autoref{fig:center_of_bounds_effect}, the problems discussed for the x-axis plot are even more prevalent.

\begin{figure}[H]
    \centering
    \includegraphics[width = 0.85\textwidth]{assets/images/center_of_bounds_line.pdf}
    \caption{Field distribution for each \gls{PHM} model along each model's the center of bounds. The number near each model is the standard deviation for the distribution. The curves were smoothed using a moving average with a window size of 50 samples.}
    \label{fig:center_of_bounds_effect}
\end{figure}

Looking at \autoref{fig:center_of_bounds_effect} it is evident that for each model the variability of the electric field is not negligible. The graph indicates the values on the line that cuts the center of bounds for each model, meaning that is passes through many different layers at different heights in the head, giving a clearer picture of what happens. 

To better understand the cause of this difference and gauge which models have similar patterns, \gls{PCA} was performed with the features as summarized in \autoref{tab:pca_features}. For each model, the thickness was calculated from the coordinate points along the x-axis line (\autoref{fig:brain_slice_for_effects_x}), in the corresponding regions. Each of the maximum modulation envelope electric field was calculated in the whole volume of each designated region on \autoref{tab:pca_features}, same as the standard deviation.

\begin{table}[!ht]
	\centering
    \caption[\gls{PCA} feature values]{Feature values used in the \gls{PCA} analysis. The electric field values, denoted with $E$ are the mean for each model on the designated region. $\sigma$ is the corresponding standard deviation for each electric field value. The thickness value is the sum of the skin, skull , and \gls{CSF} thickness along the x-axis line for each model.}
    \label{tab:pca_features}
    \begin{tabular}{|c|c|c|c|c|c|c|c|}
        % \hline
        \cline{2-8}
        \multicolumn{1}{c|}{} & \cellcolor[HTML]{C0C0C0}\textbf{[mm]} & \multicolumn{6}{c|}{\cellcolor[HTML]{C0C0C0}\textbf{[mV/m]}} \\ \hline
        \rowcolor[HTML]{C0C0C0} 
        \textbf{Model ID} & \textbf{Thickness} &  $\boldsymbol{E_{CSF}}$ & $\boldsymbol{E_{GM}}$ & $\boldsymbol{E_{WM}}$ & $\boldsymbol{\sigma_{CSF}}$ & $\boldsymbol{\sigma_{GM}}$ & $\boldsymbol{\sigma_{WM}}$ \\\hline
        103414 & 25.36 &             31.70 &            41.70 &            43.35 &            11.89 &           10.84 &            9.99 \\\hline
        105014 & 36.60 &             26.43 &            34.37 &            35.35 &            10.39 &            9.51 &            8.39 \\\hline
        105115 & 37.09 &             24.30 &            32.45 &            33.33 &             8.91 &            8.57 &            7.49 \\\hline
        110411 & 39.84 &             24.67 &            33.33 &            34.17 &             9.50 &            9.36 &            8.42 \\\hline
        111716 & 38.71 &             26.08 &            35.06 &            35.91 &            10.37 &           10.35 &            9.29 \\\hline
        113619 & 37.18 &             26.16 &            35.23 &            36.64 &            10.51 &           10.34 &            9.86 \\\hline
        117122 & 37.38 &             29.17 &            39.43 &            40.84 &            11.76 &           12.16 &           11.38 \\\hline
        163129 & 39.07 &             26.77 &            35.50 &            36.67 &            10.18 &            9.98 &            8.92 \\\hline
        196750 & 39.88 &             28.76 &            38.12 &            39.40 &            10.81 &           10.29 &            9.23 \\\hline
    \end{tabular}
\end{table}

Correlating the data from \autoref{tab:pca_features} in a \gls{PCA} chart, provides an insight on similarities or the lack thereof, between the models used. As seen in \autoref{fig:model_pca}, there is some clustering of models indicating that they more correlated than the rest, based on the provided feature set. That said, it is evident that the models 111716, 113619, and 163129 are clustered together, in a weak way, and as seen on \autoref{tab:pca_features} they have very similar thickness values. Models like 103414 and 110411 are on the 2 sides of the graph and they have very different thickness values. The two latter models have also significant difference in the mean electric field values on the white matter.

\begin{figure}[H]
    \centering
    \includegraphics[width = 0.85\textwidth]{assets/images/model_pca.pdf}
    \caption{Principle Component Analysis (PCA) graph for the different models, based on the features included in \autoref{tab:pca_features}. The explained variance ratio in $\boldsymbol{96.18\%}$.}
    \label{fig:model_pca}
\end{figure}

From the analysis provided above, it seems that there is a dependance on the thickness of the model with the final electric field sparsity and distribution, considering the same electrodes are operating for each one of the models. Despite the initial evidence presented above, further investigation with more models is required to reach a conclusion. What is certain is that there is a dependance on the model geometry and the final pattern as it can also be seen in \autoref{fig:models_pattern_variation}.

\begin{figure}[H]
    \centering
    \includegraphics[width = 0.8\textwidth]{assets/images/brain_pattern_variation_models.pdf}
    \caption{Variation of the electric field pattern distribution for the different models. The scale is the same for all models.}
    \label{fig:models_pattern_variation}
\end{figure}

In all models included in this work and all simulations, a uniform white matter and skull are assumed. This is not the correct since the white matter has different regions with different densities, and similarly the skull has the spongiform regions with very low conductivity value. As shown by \paper{Rampersad}{Rampersad2013_skull_approximations}, on \gls{tES} and specifically \gls{tDCS}, the different conductivity values for the layered bone structure \textit{(spongiform and compact bone)} of the skull can have a considerable effect if not considered in the simulations. The white matter uniformity consideration on the other hand, provides no information for the different regions and it puts a road block on assessing the impact of the electric field on the regions of interest, most of them lying in the white matter region, near the ventricles \textit{(e.g., pineal gland and globus palidus)}.

\subsection{Varying Electrode Position}
\label{subsec:varying_electrode_position}

Model anatomy is one of the crucial factors that must be considered when simulating with \gls{tTIS}. Although it is not the only player, since the current ratio between the two electrodes, and the position of them, can influence the results a lot.

As shown by \paper{Rampersad}{Rampersad2019}, the current ratio has significant impact regarding the maximum potential achieved in a given volume. This dependance is well explained as shown in \autoref{fig:mod_env_amplitude_var}, due to the fact that when the two electrodes are not placed symmetrically, the pair closest to the region of interest takes over. The resulting modulation amplitude is much lower than the one seen in \autoref{fig:modulation_showcase}.

\begin{figure}[H]
    \centering
    \includegraphics[width = \textwidth]{assets/images/modulation_envelope_2-1.pdf}
    \caption{Modulation envelope and the effect of differing amplitudes between the two fields}
    \label{fig:mod_env_amplitude_var}
\end{figure}

In this work, the attention will be shifted towards the electrode position variation and its subsequent effects. One can imagine that moving the electrodes or changing the combination used can have an impact on the final pattern. It is evident from \autoref{fig:elec_position_variation} that using the combination just one step above the previous one can have significant impact on the field distribution.

\begin{figure}[H]
    \centering
    \begin{subfigure}[b]{0.49\textwidth}
        \includegraphics[width = \textwidth]{assets/images/105014_26_16.png}
        \caption{P8, F8 and P7, F7 electrode pairs.}
        \label{fig:26_16_elec_pair}
    \end{subfigure}
    \begin{subfigure}[b]{0.49\textwidth}
        \includegraphics[width = \textwidth]{assets/images/105014_25_15.png}
        \caption{P4, F4 and P3, F3 electrode pairs.}
        \label{fig:25_15_elec_pair}
    \end{subfigure}
    \caption{Electric field distribution with varying electrode position}
    \label{fig:elec_position_variation}
\end{figure}

Even if one electrode is moved and the others stay as they are, significant differences are observed \textit{(\autoref{fig:one_elec_moved_var})}. One of the main reasons making \cref{fig:26_16_elec_pair,fig:25_15_elec_pair} differ significantly is the lack of granularity the 10-20 system provides. Limited only to 19 electrodes, there is not much room for movement, hence a system with more electrodes is required and in that case th 10-10 system can be a good candidate. The reason that the 10-5 system is not considered is mainly due to the very high number of electrodes \textit{(329 in total)}.

\begin{figure}[H]
    \centering
    \begin{subfigure}[b]{0.49\textwidth}
        \includegraphics[width = \textwidth]{assets/images/105014_26_16_22_12_z-axis.png}
        \caption{P8, F8 and P7, F7 electrode pairs.}
        \label{fig:26_16_12_22_elec_pair}
    \end{subfigure}
    \begin{subfigure}[b]{0.49\textwidth}
        \includegraphics[width = \textwidth]{assets/images/105014_26_16_23_12_z-axis.png}
        \caption{P8, F8 and P3, F7 electrode pairs.}
        \label{fig:25_15_13_24_elec_pair}
    \end{subfigure}
    \caption{Electric field distribution with varying electrode position}
    \label{fig:one_elec_moved_var}
\end{figure}

As it will be discussed in a more detail in the \nameref{sec:discussion}, the dependance on the position, the current ratio, and the model anatomy drives the need for seeking personalized optimization for each model. The optimization will be part of future work and it is not covered in the current work.

To better illustrate the granularity mentioned above, the \textbf{103414} model was meshed with all the 10-10 system electrodes and the solved with the pairs as shown in \autoref{fig:one_elec_moved_var_10-10}.

\begin{figure}[H]
    \centering
    \begin{subfigure}[b]{0.49\textwidth}
        \includegraphics[width = \textwidth]{assets/images/103414_35_68_43_76.png}
        \caption{FT7, P7 and FT8, P8 electrode pairs.}
        \label{fig:35_68_43_76_elec_pair}
    \end{subfigure}
    \begin{subfigure}[b]{0.49\textwidth}
        \includegraphics[width = \textwidth]{assets/images/103414_35_68_43_75.png}
        \caption{FT7, P7 and FT8, P6 electrode pairs.}
        \label{fig:35_68_43_75_elec_pair}
    \end{subfigure}
    \caption{Electric field distribution with varying electrode position on the 10-10 system.}
    \label{fig:one_elec_moved_var_10-10}
\end{figure}

What is clear from the figure above is that the effect the electrode position on the electric field pattern can be manipulated with more refined moves.

\pagebreak
\chapter{Discussion \& Conclusions}

% Helper sections
\newacronym{DBS}{DBS}{Deep Brain Stimulation}
\newacronym{TI}{TI}{Temporal Interference}
\newacronym{TMS}{TMS}{Transcranial Magnetic Stimulation}
\newacronym{tES}{tES}{Transcranial Electric Stimulation}
\newacronym{tACS}{tACS}{Transcranial Alternating Current Stimulation}
\newacronym{tDCS}{tDCS}{Transcranial Direct Current Stimulation}
\newacronym{tTIS}{tTIS}{Transcranial Temporal Interference Stimulation}
\newacronym{SAM}{SAM}{Simple Anthropomorphic Model}
\newacronym{EEG}{EEG}{Electroencephalography}
\newacronym{FEM}{FEM}{Finite Element Method}
\newacronym{ROI}{ROI}{Region Of Interest}
\newacronym{AM}{AM}{Amplitude Modulation}
\newacronym{FD}{FD}{Finite Difference}
\newacronym{FE}{FE}{Finite Element}
\newacronym{CSF}{CSF}{Cerebrospinal Fluid}
\newacronym{IT'IS}{IT'IS}{Information Technologies in Society}
\newacronym{CAD}{CAD}{Computer-Aided Design}
\newacronym{MRI}{MRI}{Magnetic Resonance Imaging}
\newacronym{MR}{MR}{Magnetic Resonance}
\newacronym{CT}{CT}{Computerized Tomography}
\newacronym{PHM}{PHM}{Population Head Model}
\newacronym{STL}{STL}{Stereolithography}
\newacronym{GUI}{GUI}{Graphical User Interface}
\newacronym{CGAL}{CGAL}{Computational Geometry Algorithms Library}
\newacronym{BE}{BE}{Boundary Element}
\newacronym{MATLAB}{MATLAB}{Matrix Laboratory}
\newacronym{RAM}{RAM}{Random Access Memory}
\newacronym{UMFPACK}{UMFPACK}{Unsymmetric MultiFrontal method}
\newacronym{AUTh}{AUTh}{Aristotle University of Thessaloniki}
\newacronym{IT}{IT}{Information Technology}
\newacronym{HPC}{HPC}{High Power Computing}
\newacronym{CG}{CG}{Conjugate Gradient}
\newacronym{HYPRE}{HYPRE}{High Performance Preconditioners}
\newacronym{AMG}{AMG}{Algebraic Multigrid}
\newacronym{HMIS}{HMIS}{Hybrid Modified Independent Set}
\newacronym{PETSc}{PETSc}{Portable, Extensible Toolkit for Scientific Computation}
\newacronym{VTK}{VTK}{Visualization Toolkit}
\newacronym{YAML}{YAML}{YAML Ain't Markup Language}
\newacronym{PCA}{PCA}{Principal Component Analysis}
\newacronym{UML}{UML}{Universal Modeling Language}
\newacronym{SimNIBS}{SimNIBS}{Simulation of Non-invasive Brain Stimulation}
\newacronym{ZMT}{ZMT}{Zürich MedTech}

\newglossaryentry{basal ganglia}{
    name = {basal ganglia},
    description = {a mass of nerve cell bodies (gray matter) located deep within the white matter of the cerebrum. Has connections with areas that subconsciously control movement}
}
\newglossaryentry{grey matter}{
    name = {grey matter},
    description = {superficial brain area where neuron bodies are located}
}

\newglossaryentry{tetrahedron aspect ratio}{
    name = {tetrahedron aspect ratio},
    description = {The ratio between the longest edge length and the smallest side height of each tetrahedral element in the mesh.}
}

\newglossaryentry{nasion}{
    name = {nasion},
    description = {The most anterior point where the frontal and nasal bones join (frontonasal suture)}
}

\newglossaryentry{inion}{
    name = {inion},
    description = {Highest point of the squamous part of the occipital bone}
}

\newglossaryentry{preauricular}{
    name = {preauricular},
    description = {Protuberance anterior to the auricle of the ear}
} % Input the glossary

% Print the glossary entries
\pagebreak
\printglossaries

% Print the references
\pagebreak
\printbibliography[heading=bibintoc]

\appendix
\pagebreak
\chapter{Algorithms}
\label{appndx:algorithms}

\begin{algorithm}[!ht]
\SetAlgoLined
\SetKwInOut{KwIn}{Input}
\SetKwInOut{KwOut}{Output}

\KwIn{Two lists $[e^1_{ij}]$ and $[e^2_{ij}]$, $i = 1, 2, \cdots, n$ and $j = 1, 2, 3$, where each element $i$ is a \texttt{float} of the respective electric field at the $j$ coordinate.}
\KwOut{A list of \texttt{float} size $n$ containing the maximum envelope amplitude of the electric field at each point of input.}
Set a list $[e^{am}_i]$, $i = 1, 2, \cdots, n$, of type \texttt{float} to save calculated result\;
Calculate the dot product of $[e^1_{ij}]$ and $[e^2_{ij}]$ across $j$ and save it to a list $[dot_i$], $i = 1, 2, \cdots, n$, of type \texttt{float}\;
Calculate the cross product of $[e^1_{ij}]$ and $[e^2_{ij}]$ across $j$ and save it to a list $[cross_i]$, $i = 1, 2, \cdots, n$, of type \texttt{float}\;
\ForEach{$i \in [dot_i], [cross_i]$}{
    Calculate the \textit{arctan} of $\dfrac{[cross_i]}{[dot_i]}$ and save it to a list $[angles_i]$, $i = 1, 2, \cdots, n$, of type \texttt{float}\;
}
\ForEach{$i,j \in [e^2_{ij}]$}{
    \If{$[angles_i] \geq\pi/2$}{
        $e^2_{ij} \longleftarrow -e^2_{ij}$\;
    }
}
Execute the code section from lines $\boldsymbol{2 - 6}$ again\;
$[cosangs_i] \longleftarrow \cos{[angles_i]}$\;
Set a list $[e^{minus}_{ij}]$ as $[e^1_{ij}] - [e^2_{ij}]$, for all $\forall i,j$\;
\ForEach{$i,j \in [e^2_{ij}]\;\text{across}\; j$}{
    $[mag\_e^1_i] \longleftarrow$ Magnitude of $[e^1_{ij}]$\;
    $[mag\_e^2_i] \longleftarrow$ Magnitude of $[e^2_{ij}]$\;
    \uIf{$[e^2_{ij}] < [e^1_{ij}]\cdot[cosangs_i]$}{
       $[e^{am}_i] \longleftarrow 2\cdot[e^2_{ij}]$\; 
    }
    \ElseIf{$[e^1_{ij}] > [e^2_{ij}]$}{
       $[e^{am}_i] \longleftarrow 2\dfrac{\big|\big|\vec{E_2}\big|\times\big(\vec{E_1} - \vec{E_2}\big)\big|}{\big|\vec{E_1} - \vec{E_2}\big|}$\; 
    }
    \uIf{$[e^1_{ij}] < [e^2_{ij}]\cdot[cosangs_i]$}{
       $[e^{am}_i] \longleftarrow 2\cdot[e^1_{ij}]$\; 
    }
    \ElseIf{$[e^2_{ij}] > [e^1_{ij}]$}{
       $[e^{am}_i] \longleftarrow 2\dfrac{\big|\big|\vec{E_1}\big|\times\big(\vec{E_2} - \vec{E_1}\big)\big|}{\big|\vec{E_2} - \vec{E_1}\big|}$\; 
    }
}
\caption{Calculate the maximum envelope modulation amplitude at all directions}
\label{alg:max_modulation_amplitude}
\end{algorithm}

\pagebreak

\begin{algorithm}[!ht]
\SetAlgoLined
\SetKwInOut{KwIn}{Input}
\SetKwInOut{KwOut}{Output}

\KwIn{$msh$ \gls{FE} mesh and $[init\_point_i]\;, i = 1..3$ as the initial point of the vector.}
\KwOut{The unit normal vector of the electrode surface with respect to the \gls{FE} mesh.}
$adj\_face\longleftarrow$ First adjacent face to $[init\_point_i]\in msh$\;
$[end\_point^1_i]\longleftarrow$ Vertex corresponding to $adj\_face\;\neq [init\_point_i]$\;
$[end\_point^2_i]\longleftarrow$ Vertex corresponding to $adj\_face\;\neq [init\_point_i]$ and different than $[end\_point^1_i]$\;
$[normal_i]\longleftarrow ([end\_point^1_i] - [init\_point_i])\times([end\_point^2_i] - [init\_point_i])$\;
$[normal_i]\longleftarrow [normal_i]/\big|[normal_i]\big|$\;
\caption{Calculate the normal vector coordinates}
\label{alg:electrode_orientation}
\end{algorithm}

\end{document}
